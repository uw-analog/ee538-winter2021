\documentclass[11pt]{article}

    \usepackage[breakable]{tcolorbox}
    \usepackage{parskip} % Stop auto-indenting (to mimic markdown behaviour)
    
    \usepackage{iftex}
    \ifPDFTeX
    	\usepackage[T1]{fontenc}
    	\usepackage{mathpazo}
    \else
    	\usepackage{fontspec}
    \fi

    % Basic figure setup, for now with no caption control since it's done
    % automatically by Pandoc (which extracts ![](path) syntax from Markdown).
    \usepackage{graphicx}
    % Maintain compatibility with old templates. Remove in nbconvert 6.0
    \let\Oldincludegraphics\includegraphics
    % Ensure that by default, figures have no caption (until we provide a
    % proper Figure object with a Caption API and a way to capture that
    % in the conversion process - todo).
    \usepackage{caption}
    \DeclareCaptionFormat{nocaption}{}
    \captionsetup{format=nocaption,aboveskip=0pt,belowskip=0pt}

    \usepackage[Export]{adjustbox} % Used to constrain images to a maximum size
    \adjustboxset{max size={0.9\linewidth}{0.9\paperheight}}
    \usepackage{float}
    \floatplacement{figure}{H} % forces figures to be placed at the correct location
    \usepackage{xcolor} % Allow colors to be defined
    \usepackage{enumerate} % Needed for markdown enumerations to work
    \usepackage{geometry} % Used to adjust the document margins
    \usepackage{amsmath} % Equations
    \usepackage{amssymb} % Equations
    \usepackage{textcomp} % defines textquotesingle
    % Hack from http://tex.stackexchange.com/a/47451/13684:
    \AtBeginDocument{%
        \def\PYZsq{\textquotesingle}% Upright quotes in Pygmentized code
    }
    \usepackage{upquote} % Upright quotes for verbatim code
    \usepackage{eurosym} % defines \euro
    \usepackage[mathletters]{ucs} % Extended unicode (utf-8) support
    \usepackage{fancyvrb} % verbatim replacement that allows latex
    \usepackage{grffile} % extends the file name processing of package graphics 
                         % to support a larger range
    \makeatletter % fix for grffile with XeLaTeX
    \def\Gread@@xetex#1{%
      \IfFileExists{"\Gin@base".bb}%
      {\Gread@eps{\Gin@base.bb}}%
      {\Gread@@xetex@aux#1}%
    }
    \makeatother

    % The hyperref package gives us a pdf with properly built
    % internal navigation ('pdf bookmarks' for the table of contents,
    % internal cross-reference links, web links for URLs, etc.)
    \usepackage{hyperref}
    % The default LaTeX title has an obnoxious amount of whitespace. By default,
    % titling removes some of it. It also provides customization options.
    \usepackage{titling}
    \usepackage{longtable} % longtable support required by pandoc >1.10
    \usepackage{booktabs}  % table support for pandoc > 1.12.2
    \usepackage[inline]{enumitem} % IRkernel/repr support (it uses the enumerate* environment)
    \usepackage[normalem]{ulem} % ulem is needed to support strikethroughs (\sout)
                                % normalem makes italics be italics, not underlines
    \usepackage{mathrsfs}
    

    
    % Colors for the hyperref package
    \definecolor{urlcolor}{rgb}{0,.145,.698}
    \definecolor{linkcolor}{rgb}{.71,0.21,0.01}
    \definecolor{citecolor}{rgb}{.12,.54,.11}

    % ANSI colors
    \definecolor{ansi-black}{HTML}{3E424D}
    \definecolor{ansi-black-intense}{HTML}{282C36}
    \definecolor{ansi-red}{HTML}{E75C58}
    \definecolor{ansi-red-intense}{HTML}{B22B31}
    \definecolor{ansi-green}{HTML}{00A250}
    \definecolor{ansi-green-intense}{HTML}{007427}
    \definecolor{ansi-yellow}{HTML}{DDB62B}
    \definecolor{ansi-yellow-intense}{HTML}{B27D12}
    \definecolor{ansi-blue}{HTML}{208FFB}
    \definecolor{ansi-blue-intense}{HTML}{0065CA}
    \definecolor{ansi-magenta}{HTML}{D160C4}
    \definecolor{ansi-magenta-intense}{HTML}{A03196}
    \definecolor{ansi-cyan}{HTML}{60C6C8}
    \definecolor{ansi-cyan-intense}{HTML}{258F8F}
    \definecolor{ansi-white}{HTML}{C5C1B4}
    \definecolor{ansi-white-intense}{HTML}{A1A6B2}
    \definecolor{ansi-default-inverse-fg}{HTML}{FFFFFF}
    \definecolor{ansi-default-inverse-bg}{HTML}{000000}

    % commands and environments needed by pandoc snippets
    % extracted from the output of `pandoc -s`
    \providecommand{\tightlist}{%
      \setlength{\itemsep}{0pt}\setlength{\parskip}{0pt}}
    \DefineVerbatimEnvironment{Highlighting}{Verbatim}{commandchars=\\\{\}}
    % Add ',fontsize=\small' for more characters per line
    \newenvironment{Shaded}{}{}
    \newcommand{\KeywordTok}[1]{\textcolor[rgb]{0.00,0.44,0.13}{\textbf{{#1}}}}
    \newcommand{\DataTypeTok}[1]{\textcolor[rgb]{0.56,0.13,0.00}{{#1}}}
    \newcommand{\DecValTok}[1]{\textcolor[rgb]{0.25,0.63,0.44}{{#1}}}
    \newcommand{\BaseNTok}[1]{\textcolor[rgb]{0.25,0.63,0.44}{{#1}}}
    \newcommand{\FloatTok}[1]{\textcolor[rgb]{0.25,0.63,0.44}{{#1}}}
    \newcommand{\CharTok}[1]{\textcolor[rgb]{0.25,0.44,0.63}{{#1}}}
    \newcommand{\StringTok}[1]{\textcolor[rgb]{0.25,0.44,0.63}{{#1}}}
    \newcommand{\CommentTok}[1]{\textcolor[rgb]{0.38,0.63,0.69}{\textit{{#1}}}}
    \newcommand{\OtherTok}[1]{\textcolor[rgb]{0.00,0.44,0.13}{{#1}}}
    \newcommand{\AlertTok}[1]{\textcolor[rgb]{1.00,0.00,0.00}{\textbf{{#1}}}}
    \newcommand{\FunctionTok}[1]{\textcolor[rgb]{0.02,0.16,0.49}{{#1}}}
    \newcommand{\RegionMarkerTok}[1]{{#1}}
    \newcommand{\ErrorTok}[1]{\textcolor[rgb]{1.00,0.00,0.00}{\textbf{{#1}}}}
    \newcommand{\NormalTok}[1]{{#1}}
    
    % Additional commands for more recent versions of Pandoc
    \newcommand{\ConstantTok}[1]{\textcolor[rgb]{0.53,0.00,0.00}{{#1}}}
    \newcommand{\SpecialCharTok}[1]{\textcolor[rgb]{0.25,0.44,0.63}{{#1}}}
    \newcommand{\VerbatimStringTok}[1]{\textcolor[rgb]{0.25,0.44,0.63}{{#1}}}
    \newcommand{\SpecialStringTok}[1]{\textcolor[rgb]{0.73,0.40,0.53}{{#1}}}
    \newcommand{\ImportTok}[1]{{#1}}
    \newcommand{\DocumentationTok}[1]{\textcolor[rgb]{0.73,0.13,0.13}{\textit{{#1}}}}
    \newcommand{\AnnotationTok}[1]{\textcolor[rgb]{0.38,0.63,0.69}{\textbf{\textit{{#1}}}}}
    \newcommand{\CommentVarTok}[1]{\textcolor[rgb]{0.38,0.63,0.69}{\textbf{\textit{{#1}}}}}
    \newcommand{\VariableTok}[1]{\textcolor[rgb]{0.10,0.09,0.49}{{#1}}}
    \newcommand{\ControlFlowTok}[1]{\textcolor[rgb]{0.00,0.44,0.13}{\textbf{{#1}}}}
    \newcommand{\OperatorTok}[1]{\textcolor[rgb]{0.40,0.40,0.40}{{#1}}}
    \newcommand{\BuiltInTok}[1]{{#1}}
    \newcommand{\ExtensionTok}[1]{{#1}}
    \newcommand{\PreprocessorTok}[1]{\textcolor[rgb]{0.74,0.48,0.00}{{#1}}}
    \newcommand{\AttributeTok}[1]{\textcolor[rgb]{0.49,0.56,0.16}{{#1}}}
    \newcommand{\InformationTok}[1]{\textcolor[rgb]{0.38,0.63,0.69}{\textbf{\textit{{#1}}}}}
    \newcommand{\WarningTok}[1]{\textcolor[rgb]{0.38,0.63,0.69}{\textbf{\textit{{#1}}}}}
    
    
    % Define a nice break command that doesn't care if a line doesn't already
    % exist.
    \def\br{\hspace*{\fill} \\* }
    % Math Jax compatibility definitions
    \def\gt{>}
    \def\lt{<}
    \let\Oldtex\TeX
    \let\Oldlatex\LaTeX
    \renewcommand{\TeX}{\textrm{\Oldtex}}
    \renewcommand{\LaTeX}{\textrm{\Oldlatex}}
    % Document parameters
    % Document title
    \title{2021\_02\_17\_EE538\_Lecture7\_W2021}
    
    
    
    
    
% Pygments definitions
\makeatletter
\def\PY@reset{\let\PY@it=\relax \let\PY@bf=\relax%
    \let\PY@ul=\relax \let\PY@tc=\relax%
    \let\PY@bc=\relax \let\PY@ff=\relax}
\def\PY@tok#1{\csname PY@tok@#1\endcsname}
\def\PY@toks#1+{\ifx\relax#1\empty\else%
    \PY@tok{#1}\expandafter\PY@toks\fi}
\def\PY@do#1{\PY@bc{\PY@tc{\PY@ul{%
    \PY@it{\PY@bf{\PY@ff{#1}}}}}}}
\def\PY#1#2{\PY@reset\PY@toks#1+\relax+\PY@do{#2}}

\expandafter\def\csname PY@tok@w\endcsname{\def\PY@tc##1{\textcolor[rgb]{0.73,0.73,0.73}{##1}}}
\expandafter\def\csname PY@tok@c\endcsname{\let\PY@it=\textit\def\PY@tc##1{\textcolor[rgb]{0.25,0.50,0.50}{##1}}}
\expandafter\def\csname PY@tok@cp\endcsname{\def\PY@tc##1{\textcolor[rgb]{0.74,0.48,0.00}{##1}}}
\expandafter\def\csname PY@tok@k\endcsname{\let\PY@bf=\textbf\def\PY@tc##1{\textcolor[rgb]{0.00,0.50,0.00}{##1}}}
\expandafter\def\csname PY@tok@kp\endcsname{\def\PY@tc##1{\textcolor[rgb]{0.00,0.50,0.00}{##1}}}
\expandafter\def\csname PY@tok@kt\endcsname{\def\PY@tc##1{\textcolor[rgb]{0.69,0.00,0.25}{##1}}}
\expandafter\def\csname PY@tok@o\endcsname{\def\PY@tc##1{\textcolor[rgb]{0.40,0.40,0.40}{##1}}}
\expandafter\def\csname PY@tok@ow\endcsname{\let\PY@bf=\textbf\def\PY@tc##1{\textcolor[rgb]{0.67,0.13,1.00}{##1}}}
\expandafter\def\csname PY@tok@nb\endcsname{\def\PY@tc##1{\textcolor[rgb]{0.00,0.50,0.00}{##1}}}
\expandafter\def\csname PY@tok@nf\endcsname{\def\PY@tc##1{\textcolor[rgb]{0.00,0.00,1.00}{##1}}}
\expandafter\def\csname PY@tok@nc\endcsname{\let\PY@bf=\textbf\def\PY@tc##1{\textcolor[rgb]{0.00,0.00,1.00}{##1}}}
\expandafter\def\csname PY@tok@nn\endcsname{\let\PY@bf=\textbf\def\PY@tc##1{\textcolor[rgb]{0.00,0.00,1.00}{##1}}}
\expandafter\def\csname PY@tok@ne\endcsname{\let\PY@bf=\textbf\def\PY@tc##1{\textcolor[rgb]{0.82,0.25,0.23}{##1}}}
\expandafter\def\csname PY@tok@nv\endcsname{\def\PY@tc##1{\textcolor[rgb]{0.10,0.09,0.49}{##1}}}
\expandafter\def\csname PY@tok@no\endcsname{\def\PY@tc##1{\textcolor[rgb]{0.53,0.00,0.00}{##1}}}
\expandafter\def\csname PY@tok@nl\endcsname{\def\PY@tc##1{\textcolor[rgb]{0.63,0.63,0.00}{##1}}}
\expandafter\def\csname PY@tok@ni\endcsname{\let\PY@bf=\textbf\def\PY@tc##1{\textcolor[rgb]{0.60,0.60,0.60}{##1}}}
\expandafter\def\csname PY@tok@na\endcsname{\def\PY@tc##1{\textcolor[rgb]{0.49,0.56,0.16}{##1}}}
\expandafter\def\csname PY@tok@nt\endcsname{\let\PY@bf=\textbf\def\PY@tc##1{\textcolor[rgb]{0.00,0.50,0.00}{##1}}}
\expandafter\def\csname PY@tok@nd\endcsname{\def\PY@tc##1{\textcolor[rgb]{0.67,0.13,1.00}{##1}}}
\expandafter\def\csname PY@tok@s\endcsname{\def\PY@tc##1{\textcolor[rgb]{0.73,0.13,0.13}{##1}}}
\expandafter\def\csname PY@tok@sd\endcsname{\let\PY@it=\textit\def\PY@tc##1{\textcolor[rgb]{0.73,0.13,0.13}{##1}}}
\expandafter\def\csname PY@tok@si\endcsname{\let\PY@bf=\textbf\def\PY@tc##1{\textcolor[rgb]{0.73,0.40,0.53}{##1}}}
\expandafter\def\csname PY@tok@se\endcsname{\let\PY@bf=\textbf\def\PY@tc##1{\textcolor[rgb]{0.73,0.40,0.13}{##1}}}
\expandafter\def\csname PY@tok@sr\endcsname{\def\PY@tc##1{\textcolor[rgb]{0.73,0.40,0.53}{##1}}}
\expandafter\def\csname PY@tok@ss\endcsname{\def\PY@tc##1{\textcolor[rgb]{0.10,0.09,0.49}{##1}}}
\expandafter\def\csname PY@tok@sx\endcsname{\def\PY@tc##1{\textcolor[rgb]{0.00,0.50,0.00}{##1}}}
\expandafter\def\csname PY@tok@m\endcsname{\def\PY@tc##1{\textcolor[rgb]{0.40,0.40,0.40}{##1}}}
\expandafter\def\csname PY@tok@gh\endcsname{\let\PY@bf=\textbf\def\PY@tc##1{\textcolor[rgb]{0.00,0.00,0.50}{##1}}}
\expandafter\def\csname PY@tok@gu\endcsname{\let\PY@bf=\textbf\def\PY@tc##1{\textcolor[rgb]{0.50,0.00,0.50}{##1}}}
\expandafter\def\csname PY@tok@gd\endcsname{\def\PY@tc##1{\textcolor[rgb]{0.63,0.00,0.00}{##1}}}
\expandafter\def\csname PY@tok@gi\endcsname{\def\PY@tc##1{\textcolor[rgb]{0.00,0.63,0.00}{##1}}}
\expandafter\def\csname PY@tok@gr\endcsname{\def\PY@tc##1{\textcolor[rgb]{1.00,0.00,0.00}{##1}}}
\expandafter\def\csname PY@tok@ge\endcsname{\let\PY@it=\textit}
\expandafter\def\csname PY@tok@gs\endcsname{\let\PY@bf=\textbf}
\expandafter\def\csname PY@tok@gp\endcsname{\let\PY@bf=\textbf\def\PY@tc##1{\textcolor[rgb]{0.00,0.00,0.50}{##1}}}
\expandafter\def\csname PY@tok@go\endcsname{\def\PY@tc##1{\textcolor[rgb]{0.53,0.53,0.53}{##1}}}
\expandafter\def\csname PY@tok@gt\endcsname{\def\PY@tc##1{\textcolor[rgb]{0.00,0.27,0.87}{##1}}}
\expandafter\def\csname PY@tok@err\endcsname{\def\PY@bc##1{\setlength{\fboxsep}{0pt}\fcolorbox[rgb]{1.00,0.00,0.00}{1,1,1}{\strut ##1}}}
\expandafter\def\csname PY@tok@kc\endcsname{\let\PY@bf=\textbf\def\PY@tc##1{\textcolor[rgb]{0.00,0.50,0.00}{##1}}}
\expandafter\def\csname PY@tok@kd\endcsname{\let\PY@bf=\textbf\def\PY@tc##1{\textcolor[rgb]{0.00,0.50,0.00}{##1}}}
\expandafter\def\csname PY@tok@kn\endcsname{\let\PY@bf=\textbf\def\PY@tc##1{\textcolor[rgb]{0.00,0.50,0.00}{##1}}}
\expandafter\def\csname PY@tok@kr\endcsname{\let\PY@bf=\textbf\def\PY@tc##1{\textcolor[rgb]{0.00,0.50,0.00}{##1}}}
\expandafter\def\csname PY@tok@bp\endcsname{\def\PY@tc##1{\textcolor[rgb]{0.00,0.50,0.00}{##1}}}
\expandafter\def\csname PY@tok@fm\endcsname{\def\PY@tc##1{\textcolor[rgb]{0.00,0.00,1.00}{##1}}}
\expandafter\def\csname PY@tok@vc\endcsname{\def\PY@tc##1{\textcolor[rgb]{0.10,0.09,0.49}{##1}}}
\expandafter\def\csname PY@tok@vg\endcsname{\def\PY@tc##1{\textcolor[rgb]{0.10,0.09,0.49}{##1}}}
\expandafter\def\csname PY@tok@vi\endcsname{\def\PY@tc##1{\textcolor[rgb]{0.10,0.09,0.49}{##1}}}
\expandafter\def\csname PY@tok@vm\endcsname{\def\PY@tc##1{\textcolor[rgb]{0.10,0.09,0.49}{##1}}}
\expandafter\def\csname PY@tok@sa\endcsname{\def\PY@tc##1{\textcolor[rgb]{0.73,0.13,0.13}{##1}}}
\expandafter\def\csname PY@tok@sb\endcsname{\def\PY@tc##1{\textcolor[rgb]{0.73,0.13,0.13}{##1}}}
\expandafter\def\csname PY@tok@sc\endcsname{\def\PY@tc##1{\textcolor[rgb]{0.73,0.13,0.13}{##1}}}
\expandafter\def\csname PY@tok@dl\endcsname{\def\PY@tc##1{\textcolor[rgb]{0.73,0.13,0.13}{##1}}}
\expandafter\def\csname PY@tok@s2\endcsname{\def\PY@tc##1{\textcolor[rgb]{0.73,0.13,0.13}{##1}}}
\expandafter\def\csname PY@tok@sh\endcsname{\def\PY@tc##1{\textcolor[rgb]{0.73,0.13,0.13}{##1}}}
\expandafter\def\csname PY@tok@s1\endcsname{\def\PY@tc##1{\textcolor[rgb]{0.73,0.13,0.13}{##1}}}
\expandafter\def\csname PY@tok@mb\endcsname{\def\PY@tc##1{\textcolor[rgb]{0.40,0.40,0.40}{##1}}}
\expandafter\def\csname PY@tok@mf\endcsname{\def\PY@tc##1{\textcolor[rgb]{0.40,0.40,0.40}{##1}}}
\expandafter\def\csname PY@tok@mh\endcsname{\def\PY@tc##1{\textcolor[rgb]{0.40,0.40,0.40}{##1}}}
\expandafter\def\csname PY@tok@mi\endcsname{\def\PY@tc##1{\textcolor[rgb]{0.40,0.40,0.40}{##1}}}
\expandafter\def\csname PY@tok@il\endcsname{\def\PY@tc##1{\textcolor[rgb]{0.40,0.40,0.40}{##1}}}
\expandafter\def\csname PY@tok@mo\endcsname{\def\PY@tc##1{\textcolor[rgb]{0.40,0.40,0.40}{##1}}}
\expandafter\def\csname PY@tok@ch\endcsname{\let\PY@it=\textit\def\PY@tc##1{\textcolor[rgb]{0.25,0.50,0.50}{##1}}}
\expandafter\def\csname PY@tok@cm\endcsname{\let\PY@it=\textit\def\PY@tc##1{\textcolor[rgb]{0.25,0.50,0.50}{##1}}}
\expandafter\def\csname PY@tok@cpf\endcsname{\let\PY@it=\textit\def\PY@tc##1{\textcolor[rgb]{0.25,0.50,0.50}{##1}}}
\expandafter\def\csname PY@tok@c1\endcsname{\let\PY@it=\textit\def\PY@tc##1{\textcolor[rgb]{0.25,0.50,0.50}{##1}}}
\expandafter\def\csname PY@tok@cs\endcsname{\let\PY@it=\textit\def\PY@tc##1{\textcolor[rgb]{0.25,0.50,0.50}{##1}}}

\def\PYZbs{\char`\\}
\def\PYZus{\char`\_}
\def\PYZob{\char`\{}
\def\PYZcb{\char`\}}
\def\PYZca{\char`\^}
\def\PYZam{\char`\&}
\def\PYZlt{\char`\<}
\def\PYZgt{\char`\>}
\def\PYZsh{\char`\#}
\def\PYZpc{\char`\%}
\def\PYZdl{\char`\$}
\def\PYZhy{\char`\-}
\def\PYZsq{\char`\'}
\def\PYZdq{\char`\"}
\def\PYZti{\char`\~}
% for compatibility with earlier versions
\def\PYZat{@}
\def\PYZlb{[}
\def\PYZrb{]}
\makeatother


    % For linebreaks inside Verbatim environment from package fancyvrb. 
    \makeatletter
        \newbox\Wrappedcontinuationbox 
        \newbox\Wrappedvisiblespacebox 
        \newcommand*\Wrappedvisiblespace {\textcolor{red}{\textvisiblespace}} 
        \newcommand*\Wrappedcontinuationsymbol {\textcolor{red}{\llap{\tiny$\m@th\hookrightarrow$}}} 
        \newcommand*\Wrappedcontinuationindent {3ex } 
        \newcommand*\Wrappedafterbreak {\kern\Wrappedcontinuationindent\copy\Wrappedcontinuationbox} 
        % Take advantage of the already applied Pygments mark-up to insert 
        % potential linebreaks for TeX processing. 
        %        {, <, #, %, $, ' and ": go to next line. 
        %        _, }, ^, &, >, - and ~: stay at end of broken line. 
        % Use of \textquotesingle for straight quote. 
        \newcommand*\Wrappedbreaksatspecials {% 
            \def\PYGZus{\discretionary{\char`\_}{\Wrappedafterbreak}{\char`\_}}% 
            \def\PYGZob{\discretionary{}{\Wrappedafterbreak\char`\{}{\char`\{}}% 
            \def\PYGZcb{\discretionary{\char`\}}{\Wrappedafterbreak}{\char`\}}}% 
            \def\PYGZca{\discretionary{\char`\^}{\Wrappedafterbreak}{\char`\^}}% 
            \def\PYGZam{\discretionary{\char`\&}{\Wrappedafterbreak}{\char`\&}}% 
            \def\PYGZlt{\discretionary{}{\Wrappedafterbreak\char`\<}{\char`\<}}% 
            \def\PYGZgt{\discretionary{\char`\>}{\Wrappedafterbreak}{\char`\>}}% 
            \def\PYGZsh{\discretionary{}{\Wrappedafterbreak\char`\#}{\char`\#}}% 
            \def\PYGZpc{\discretionary{}{\Wrappedafterbreak\char`\%}{\char`\%}}% 
            \def\PYGZdl{\discretionary{}{\Wrappedafterbreak\char`\$}{\char`\$}}% 
            \def\PYGZhy{\discretionary{\char`\-}{\Wrappedafterbreak}{\char`\-}}% 
            \def\PYGZsq{\discretionary{}{\Wrappedafterbreak\textquotesingle}{\textquotesingle}}% 
            \def\PYGZdq{\discretionary{}{\Wrappedafterbreak\char`\"}{\char`\"}}% 
            \def\PYGZti{\discretionary{\char`\~}{\Wrappedafterbreak}{\char`\~}}% 
        } 
        % Some characters . , ; ? ! / are not pygmentized. 
        % This macro makes them "active" and they will insert potential linebreaks 
        \newcommand*\Wrappedbreaksatpunct {% 
            \lccode`\~`\.\lowercase{\def~}{\discretionary{\hbox{\char`\.}}{\Wrappedafterbreak}{\hbox{\char`\.}}}% 
            \lccode`\~`\,\lowercase{\def~}{\discretionary{\hbox{\char`\,}}{\Wrappedafterbreak}{\hbox{\char`\,}}}% 
            \lccode`\~`\;\lowercase{\def~}{\discretionary{\hbox{\char`\;}}{\Wrappedafterbreak}{\hbox{\char`\;}}}% 
            \lccode`\~`\:\lowercase{\def~}{\discretionary{\hbox{\char`\:}}{\Wrappedafterbreak}{\hbox{\char`\:}}}% 
            \lccode`\~`\?\lowercase{\def~}{\discretionary{\hbox{\char`\?}}{\Wrappedafterbreak}{\hbox{\char`\?}}}% 
            \lccode`\~`\!\lowercase{\def~}{\discretionary{\hbox{\char`\!}}{\Wrappedafterbreak}{\hbox{\char`\!}}}% 
            \lccode`\~`\/\lowercase{\def~}{\discretionary{\hbox{\char`\/}}{\Wrappedafterbreak}{\hbox{\char`\/}}}% 
            \catcode`\.\active
            \catcode`\,\active 
            \catcode`\;\active
            \catcode`\:\active
            \catcode`\?\active
            \catcode`\!\active
            \catcode`\/\active 
            \lccode`\~`\~ 	
        }
    \makeatother

    \let\OriginalVerbatim=\Verbatim
    \makeatletter
    \renewcommand{\Verbatim}[1][1]{%
        %\parskip\z@skip
        \sbox\Wrappedcontinuationbox {\Wrappedcontinuationsymbol}%
        \sbox\Wrappedvisiblespacebox {\FV@SetupFont\Wrappedvisiblespace}%
        \def\FancyVerbFormatLine ##1{\hsize\linewidth
            \vtop{\raggedright\hyphenpenalty\z@\exhyphenpenalty\z@
                \doublehyphendemerits\z@\finalhyphendemerits\z@
                \strut ##1\strut}%
        }%
        % If the linebreak is at a space, the latter will be displayed as visible
        % space at end of first line, and a continuation symbol starts next line.
        % Stretch/shrink are however usually zero for typewriter font.
        \def\FV@Space {%
            \nobreak\hskip\z@ plus\fontdimen3\font minus\fontdimen4\font
            \discretionary{\copy\Wrappedvisiblespacebox}{\Wrappedafterbreak}
            {\kern\fontdimen2\font}%
        }%
        
        % Allow breaks at special characters using \PYG... macros.
        \Wrappedbreaksatspecials
        % Breaks at punctuation characters . , ; ? ! and / need catcode=\active 	
        \OriginalVerbatim[#1,codes*=\Wrappedbreaksatpunct]%
    }
    \makeatother

    % Exact colors from NB
    \definecolor{incolor}{HTML}{303F9F}
    \definecolor{outcolor}{HTML}{D84315}
    \definecolor{cellborder}{HTML}{CFCFCF}
    \definecolor{cellbackground}{HTML}{F7F7F7}
    
    % prompt
    \makeatletter
    \newcommand{\boxspacing}{\kern\kvtcb@left@rule\kern\kvtcb@boxsep}
    \makeatother
    \newcommand{\prompt}[4]{
        \ttfamily\llap{{\color{#2}[#3]:\hspace{3pt}#4}}\vspace{-\baselineskip}
    }
    

    
    % Prevent overflowing lines due to hard-to-break entities
    \sloppy 
    % Setup hyperref package
    \hypersetup{
      breaklinks=true,  % so long urls are correctly broken across lines
      colorlinks=true,
      urlcolor=urlcolor,
      linkcolor=linkcolor,
      citecolor=citecolor,
      }
    % Slightly bigger margins than the latex defaults
    
    \geometry{verbose,tmargin=1in,bmargin=1in,lmargin=1in,rmargin=1in}
    
    

\begin{document}
    
    \maketitle
    
    

    
    \hypertarget{ee-538-analog-integrated-circuit-design}{%
\section{EE 538: Analog Integrated Circuit
Design}\label{ee-538-analog-integrated-circuit-design}}

\hypertarget{winter-2021}{%
\subsection{Winter 2021}\label{winter-2021}}

\hypertarget{instructor-jason-silver}{%
\subsection{Instructor: Jason Silver}\label{instructor-jason-silver}}

    \hypertarget{python-packagesmodules}{%
\subsection{Python packages/modules}\label{python-packagesmodules}}

    \begin{tcolorbox}[breakable, size=fbox, boxrule=1pt, pad at break*=1mm,colback=cellbackground, colframe=cellborder]
\prompt{In}{incolor}{73}{\boxspacing}
\begin{Verbatim}[commandchars=\\\{\}]
\PY{k+kn}{import} \PY{n+nn}{matplotlib} \PY{k}{as} \PY{n+nn}{mpl}
\PY{k+kn}{from} \PY{n+nn}{matplotlib} \PY{k+kn}{import} \PY{n}{pyplot} \PY{k}{as} \PY{n}{plt}
\PY{k+kn}{import} \PY{n+nn}{numpy} \PY{k}{as} \PY{n+nn}{np}
\PY{k+kn}{from} \PY{n+nn}{scipy} \PY{k+kn}{import} \PY{n}{signal}
\PY{c+c1}{\PYZsh{}\PYZpc{}matplotlib notebook}

\PY{n}{mpl}\PY{o}{.}\PY{n}{rcParams}\PY{p}{[}\PY{l+s+s1}{\PYZsq{}}\PY{l+s+s1}{font.size}\PY{l+s+s1}{\PYZsq{}}\PY{p}{]} \PY{o}{=} \PY{l+m+mi}{12}
\PY{n}{mpl}\PY{o}{.}\PY{n}{rcParams}\PY{p}{[}\PY{l+s+s1}{\PYZsq{}}\PY{l+s+s1}{legend.fontsize}\PY{l+s+s1}{\PYZsq{}}\PY{p}{]} \PY{o}{=} \PY{l+s+s1}{\PYZsq{}}\PY{l+s+s1}{large}\PY{l+s+s1}{\PYZsq{}}

\PY{k}{def} \PY{n+nf}{plot\PYZus{}xy}\PY{p}{(}\PY{n}{x}\PY{p}{,} \PY{n}{y}\PY{p}{,} \PY{n}{xlabel}\PY{p}{,} \PY{n}{ylabel}\PY{p}{)}\PY{p}{:}
    \PY{n}{fig}\PY{p}{,} \PY{n}{ax} \PY{o}{=} \PY{n}{plt}\PY{o}{.}\PY{n}{subplots}\PY{p}{(}\PY{n}{figsize}\PY{o}{=}\PY{p}{(}\PY{l+m+mf}{10.0}\PY{p}{,} \PY{l+m+mf}{7.5}\PY{p}{)}\PY{p}{)}\PY{p}{;}
    \PY{n}{ax}\PY{o}{.}\PY{n}{plot}\PY{p}{(}\PY{n}{x}\PY{p}{,} \PY{n}{y}\PY{p}{,} \PY{l+s+s1}{\PYZsq{}}\PY{l+s+s1}{b}\PY{l+s+s1}{\PYZsq{}}\PY{p}{)}\PY{p}{;}
    \PY{n}{ax}\PY{o}{.}\PY{n}{grid}\PY{p}{(}\PY{p}{)}\PY{p}{;}
    \PY{n}{ax}\PY{o}{.}\PY{n}{set\PYZus{}xlabel}\PY{p}{(}\PY{n}{xlabel}\PY{p}{)}\PY{p}{;}
    \PY{n}{ax}\PY{o}{.}\PY{n}{set\PYZus{}ylabel}\PY{p}{(}\PY{n}{ylabel}\PY{p}{)}\PY{p}{;}
    
\PY{k}{def} \PY{n+nf}{plot\PYZus{}xy2}\PY{p}{(}\PY{n}{x1}\PY{p}{,} \PY{n}{y1}\PY{p}{,} \PY{n}{x1label}\PY{p}{,} \PY{n}{y1label}\PY{p}{,} \PY{n}{x2}\PY{p}{,} \PY{n}{y2}\PY{p}{,} \PY{n}{x2label}\PY{p}{,} \PY{n}{y2label}\PY{p}{)}\PY{p}{:}
    \PY{n}{fig}\PY{p}{,} \PY{n}{ax} \PY{o}{=} \PY{n}{plt}\PY{o}{.}\PY{n}{subplots}\PY{p}{(}\PY{l+m+mi}{2}\PY{p}{,} \PY{n}{figsize} \PY{o}{=} \PY{p}{(}\PY{l+m+mf}{10.0}\PY{p}{,} \PY{l+m+mf}{7.5}\PY{p}{)}\PY{p}{)}\PY{p}{;}
    \PY{n}{ax}\PY{p}{[}\PY{l+m+mi}{0}\PY{p}{]}\PY{o}{.}\PY{n}{plot}\PY{p}{(}\PY{n}{x1}\PY{p}{,} \PY{n}{y1}\PY{p}{,} \PY{l+s+s1}{\PYZsq{}}\PY{l+s+s1}{b}\PY{l+s+s1}{\PYZsq{}}\PY{p}{)}\PY{p}{;}
    \PY{n}{ax}\PY{p}{[}\PY{l+m+mi}{0}\PY{p}{]}\PY{o}{.}\PY{n}{set\PYZus{}ylabel}\PY{p}{(}\PY{n}{y1label}\PY{p}{)}
    \PY{n}{ax}\PY{p}{[}\PY{l+m+mi}{0}\PY{p}{]}\PY{o}{.}\PY{n}{grid}\PY{p}{(}\PY{p}{)}
    
    \PY{n}{ax}\PY{p}{[}\PY{l+m+mi}{1}\PY{p}{]}\PY{o}{.}\PY{n}{plot}\PY{p}{(}\PY{n}{x2}\PY{p}{,} \PY{n}{y2}\PY{p}{,} \PY{l+s+s1}{\PYZsq{}}\PY{l+s+s1}{b}\PY{l+s+s1}{\PYZsq{}}\PY{p}{)}\PY{p}{;}
    \PY{n}{ax}\PY{p}{[}\PY{l+m+mi}{1}\PY{p}{]}\PY{o}{.}\PY{n}{set\PYZus{}xlabel}\PY{p}{(}\PY{n}{x1label}\PY{p}{)}
    \PY{n}{ax}\PY{p}{[}\PY{l+m+mi}{1}\PY{p}{]}\PY{o}{.}\PY{n}{set\PYZus{}xlabel}\PY{p}{(}\PY{n}{x2label}\PY{p}{)}\PY{p}{;}
    \PY{n}{ax}\PY{p}{[}\PY{l+m+mi}{1}\PY{p}{]}\PY{o}{.}\PY{n}{set\PYZus{}ylabel}\PY{p}{(}\PY{n}{y2label}\PY{p}{)}\PY{p}{;}
    \PY{n}{ax}\PY{p}{[}\PY{l+m+mi}{1}\PY{p}{]}\PY{o}{.}\PY{n}{grid}\PY{p}{(}\PY{p}{)}\PY{p}{;}
    
    \PY{n}{fig}\PY{o}{.}\PY{n}{align\PYZus{}ylabels}\PY{p}{(}\PY{n}{ax}\PY{p}{[}\PY{p}{:}\PY{p}{]}\PY{p}{)}
    
\PY{k}{def} \PY{n+nf}{plot\PYZus{}x2y}\PY{p}{(}\PY{n}{x}\PY{p}{,} \PY{n}{y1}\PY{p}{,} \PY{n}{y2}\PY{p}{,} \PY{n}{xlabel}\PY{p}{,} \PY{n}{ylabel}\PY{p}{,} \PY{n}{y1label}\PY{p}{,} \PY{n}{y2label}\PY{p}{)}\PY{p}{:}
        
    \PY{n}{fig}\PY{p}{,} \PY{n}{ax} \PY{o}{=} \PY{n}{plt}\PY{o}{.}\PY{n}{subplots}\PY{p}{(}\PY{n}{figsize}\PY{o}{=}\PY{p}{(}\PY{l+m+mf}{10.0}\PY{p}{,} \PY{l+m+mf}{7.5}\PY{p}{)}\PY{p}{)}\PY{p}{;}
    \PY{n}{ax}\PY{o}{.}\PY{n}{plot}\PY{p}{(}\PY{n}{x}\PY{p}{,} \PY{n}{y1}\PY{p}{,} \PY{l+s+s1}{\PYZsq{}}\PY{l+s+s1}{b}\PY{l+s+s1}{\PYZsq{}}\PY{p}{)}
    \PY{n}{ax}\PY{o}{.}\PY{n}{plot}\PY{p}{(}\PY{n}{x}\PY{p}{,} \PY{n}{y2}\PY{p}{,} \PY{l+s+s1}{\PYZsq{}}\PY{l+s+s1}{r}\PY{l+s+s1}{\PYZsq{}}\PY{p}{)}
    \PY{n}{ax}\PY{o}{.}\PY{n}{legend}\PY{p}{(} \PY{p}{[}\PY{n}{y1label}\PY{p}{,}\PY{n}{y2label}\PY{p}{]} \PY{p}{,}\PY{n}{loc}\PY{o}{=}\PY{l+s+s1}{\PYZsq{}}\PY{l+s+s1}{upper center}\PY{l+s+s1}{\PYZsq{}}\PY{p}{,} \PY{n}{ncol}\PY{o}{=}\PY{l+m+mi}{5}\PY{p}{,} \PY{n}{fancybox}\PY{o}{=}\PY{k+kc}{True}\PY{p}{,} 
           \PY{n}{shadow}\PY{o}{=}\PY{k+kc}{True}\PY{p}{,} \PY{n}{bbox\PYZus{}to\PYZus{}anchor}\PY{o}{=}\PY{p}{(}\PY{l+m+mf}{0.5}\PY{p}{,}\PY{l+m+mf}{1.1}\PY{p}{)}\PY{p}{)}  
    \PY{n}{ax}\PY{o}{.}\PY{n}{grid}\PY{p}{(}\PY{p}{)}
    \PY{n}{ax}\PY{o}{.}\PY{n}{set\PYZus{}xlabel}\PY{p}{(}\PY{n}{xlabel}\PY{p}{)}
    \PY{n}{ax}\PY{o}{.}\PY{n}{set\PYZus{}ylabel}\PY{p}{(}\PY{n}{ylabel}\PY{p}{)}
    
\PY{k}{def} \PY{n+nf}{plot\PYZus{}xy3}\PY{p}{(}\PY{n}{x}\PY{p}{,} \PY{n}{y1}\PY{p}{,} \PY{n}{y2}\PY{p}{,} \PY{n}{y3}\PY{p}{,} \PY{n}{xlabel}\PY{p}{,} \PY{n}{y1label}\PY{p}{,} \PY{n}{y2label}\PY{p}{,} \PY{n}{y3label}\PY{p}{)}\PY{p}{:}
    \PY{n}{fig}\PY{p}{,} \PY{n}{ax} \PY{o}{=} \PY{n}{plt}\PY{o}{.}\PY{n}{subplots}\PY{p}{(}\PY{l+m+mi}{3}\PY{p}{,} \PY{n}{figsize}\PY{o}{=}\PY{p}{(}\PY{l+m+mf}{10.0}\PY{p}{,}\PY{l+m+mf}{7.5}\PY{p}{)}\PY{p}{)}
    
    \PY{n}{ax}\PY{p}{[}\PY{l+m+mi}{0}\PY{p}{]}\PY{o}{.}\PY{n}{plot}\PY{p}{(}\PY{n}{x}\PY{p}{,} \PY{n}{y1}\PY{p}{)}
    \PY{n}{ax}\PY{p}{[}\PY{l+m+mi}{0}\PY{p}{]}\PY{o}{.}\PY{n}{set\PYZus{}ylabel}\PY{p}{(}\PY{n}{y1label}\PY{p}{)}
    \PY{n}{ax}\PY{p}{[}\PY{l+m+mi}{0}\PY{p}{]}\PY{o}{.}\PY{n}{grid}\PY{p}{(}\PY{p}{)}
    
    \PY{n}{ax}\PY{p}{[}\PY{l+m+mi}{1}\PY{p}{]}\PY{o}{.}\PY{n}{plot}\PY{p}{(}\PY{n}{x}\PY{p}{,} \PY{n}{y2}\PY{p}{)}
    \PY{n}{ax}\PY{p}{[}\PY{l+m+mi}{1}\PY{p}{]}\PY{o}{.}\PY{n}{set\PYZus{}ylabel}\PY{p}{(}\PY{n}{y2label}\PY{p}{)}
    \PY{n}{ax}\PY{p}{[}\PY{l+m+mi}{1}\PY{p}{]}\PY{o}{.}\PY{n}{grid}\PY{p}{(}\PY{p}{)}
    
    \PY{n}{ax}\PY{p}{[}\PY{l+m+mi}{2}\PY{p}{]}\PY{o}{.}\PY{n}{plot}\PY{p}{(}\PY{n}{x}\PY{p}{,} \PY{n}{y3}\PY{p}{)}  
    \PY{n}{ax}\PY{p}{[}\PY{l+m+mi}{2}\PY{p}{]}\PY{o}{.}\PY{n}{set\PYZus{}ylabel}\PY{p}{(}\PY{n}{y3label}\PY{p}{)}
    \PY{n}{ax}\PY{p}{[}\PY{l+m+mi}{2}\PY{p}{]}\PY{o}{.}\PY{n}{set\PYZus{}xlabel}\PY{p}{(}\PY{n}{xlabel}\PY{p}{)}
    \PY{n}{ax}\PY{p}{[}\PY{l+m+mi}{2}\PY{p}{]}\PY{o}{.}\PY{n}{grid}\PY{p}{(}\PY{p}{)}
    
\PY{k}{def} \PY{n+nf}{plot\PYZus{}xlogy}\PY{p}{(}\PY{n}{x}\PY{p}{,} \PY{n}{y}\PY{p}{,} \PY{n}{xlabel}\PY{p}{,} \PY{n}{ylabel}\PY{p}{)}\PY{p}{:}
    \PY{n}{fig}\PY{p}{,} \PY{n}{ax} \PY{o}{=} \PY{n}{plt}\PY{o}{.}\PY{n}{subplots}\PY{p}{(}\PY{n}{figsize}\PY{o}{=}\PY{p}{(}\PY{l+m+mf}{10.0}\PY{p}{,} \PY{l+m+mf}{7.5}\PY{p}{)}\PY{p}{)}\PY{p}{;}
    \PY{n}{ax}\PY{o}{.}\PY{n}{semilogy}\PY{p}{(}\PY{n}{x}\PY{p}{,} \PY{n}{y}\PY{p}{,} \PY{l+s+s1}{\PYZsq{}}\PY{l+s+s1}{b}\PY{l+s+s1}{\PYZsq{}}\PY{p}{)}\PY{p}{;}
    \PY{n}{ax}\PY{o}{.}\PY{n}{grid}\PY{p}{(}\PY{p}{)}\PY{p}{;}
    \PY{n}{ax}\PY{o}{.}\PY{n}{set\PYZus{}xlabel}\PY{p}{(}\PY{n}{xlabel}\PY{p}{)}\PY{p}{;}
    \PY{n}{ax}\PY{o}{.}\PY{n}{set\PYZus{}ylabel}\PY{p}{(}\PY{n}{ylabel}\PY{p}{)}\PY{p}{;}
    
\PY{k}{def} \PY{n+nf}{nmos\PYZus{}iv\PYZus{}sweep}\PY{p}{(}\PY{n}{V\PYZus{}gs}\PY{p}{,} \PY{n}{V\PYZus{}ds}\PY{p}{,} \PY{n}{W}\PY{p}{,} \PY{n}{L}\PY{p}{,} \PY{n}{lmda}\PY{p}{)}\PY{p}{:}
    \PY{n}{u\PYZus{}n} \PY{o}{=} \PY{l+m+mi}{350}                 \PY{c+c1}{\PYZsh{} electron mobility (device parameter)}
    \PY{n}{e\PYZus{}ox} \PY{o}{=} \PY{l+m+mf}{3.9}\PY{o}{*}\PY{l+m+mf}{8.854e\PYZhy{}12}\PY{o}{/}\PY{l+m+mi}{100}\PY{p}{;} \PY{c+c1}{\PYZsh{} relative permittivity}
    \PY{n}{t\PYZus{}ox} \PY{o}{=} \PY{l+m+mf}{9e\PYZhy{}9}\PY{o}{*}\PY{l+m+mi}{100}\PY{p}{;}          \PY{c+c1}{\PYZsh{} oxide thickness}
    \PY{n}{C\PYZus{}ox} \PY{o}{=} \PY{n}{e\PYZus{}ox}\PY{o}{/}\PY{n}{t\PYZus{}ox}          \PY{c+c1}{\PYZsh{} oxide capacitance}
    \PY{n}{V\PYZus{}thn} \PY{o}{=} \PY{l+m+mf}{0.7}                \PY{c+c1}{\PYZsh{} threshold voltage (device parameter)}
    \PY{n}{V\PYZus{}ov} \PY{o}{=} \PY{n}{V\PYZus{}gs} \PY{o}{\PYZhy{}} \PY{n}{V\PYZus{}thn}
    \PY{n}{Ldn} \PY{o}{=} \PY{l+m+mf}{0.08e\PYZhy{}6}
    \PY{n}{Leff} \PY{o}{=} \PY{n}{L} \PY{o}{\PYZhy{}} \PY{l+m+mi}{2}\PY{o}{*}\PY{n}{Ldn}
    
    \PY{n}{I\PYZus{}d} \PY{o}{=} \PY{p}{[}\PY{p}{]}
    
    \PY{k}{for} \PY{n}{i} \PY{o+ow}{in} \PY{n+nb}{range}\PY{p}{(}\PY{n+nb}{len}\PY{p}{(}\PY{n}{V\PYZus{}ds}\PY{p}{)}\PY{p}{)}\PY{p}{:}
        \PY{n}{I\PYZus{}d}\PY{o}{.}\PY{n}{append}\PY{p}{(}\PY{n}{np}\PY{o}{.}\PY{n}{piecewise}\PY{p}{(}\PY{n}{V\PYZus{}ds}\PY{p}{[}\PY{n}{i}\PY{p}{]}\PY{p}{,} \PY{p}{[}\PY{n}{V\PYZus{}ds}\PY{p}{[}\PY{n}{i}\PY{p}{]} \PY{o}{\PYZlt{}} \PY{n}{V\PYZus{}ov}\PY{p}{,} \PY{n}{V\PYZus{}ds}\PY{p}{[}\PY{n}{i}\PY{p}{]} \PY{o}{\PYZgt{}}\PY{o}{=} \PY{n}{V\PYZus{}ov}\PY{p}{]}\PY{p}{,}
                       \PY{p}{[}\PY{n}{u\PYZus{}n}\PY{o}{*}\PY{n}{C\PYZus{}ox}\PY{o}{*}\PY{p}{(}\PY{n}{W}\PY{o}{/}\PY{n}{Leff}\PY{p}{)}\PY{o}{*}\PY{p}{(}\PY{n}{V\PYZus{}gs} \PY{o}{\PYZhy{}} \PY{n}{V\PYZus{}thn} \PY{o}{\PYZhy{}} \PY{n}{V\PYZus{}ds}\PY{p}{[}\PY{n}{i}\PY{p}{]}\PY{o}{/}\PY{l+m+mi}{2}\PY{p}{)}\PY{o}{*}\PY{n}{V\PYZus{}ds}\PY{p}{[}\PY{n}{i}\PY{p}{]}\PY{o}{*}\PY{p}{(}\PY{l+m+mi}{1}\PY{o}{+}\PY{n}{lmda}\PY{o}{*}\PY{n}{V\PYZus{}ds}\PY{p}{[}\PY{n}{i}\PY{p}{]}\PY{p}{)} \PY{p}{,} 
                        \PY{l+m+mf}{0.5}\PY{o}{*}\PY{n}{u\PYZus{}n}\PY{o}{*}\PY{n}{C\PYZus{}ox}\PY{o}{*}\PY{p}{(}\PY{n}{W}\PY{o}{/}\PY{n}{Leff}\PY{p}{)}\PY{o}{*}\PY{p}{(}\PY{n}{V\PYZus{}gs} \PY{o}{\PYZhy{}} \PY{n}{V\PYZus{}thn}\PY{p}{)}\PY{o}{*}\PY{o}{*}\PY{l+m+mi}{2}\PY{o}{*}\PY{p}{(}\PY{l+m+mi}{1}\PY{o}{+}\PY{n}{lmda}\PY{o}{*}\PY{n}{V\PYZus{}ds}\PY{p}{[}\PY{n}{i}\PY{p}{]}\PY{p}{)}\PY{p}{]}\PY{p}{)}\PY{p}{)} 
    
    \PY{k}{return} \PY{n}{np}\PY{o}{.}\PY{n}{array}\PY{p}{(}\PY{n}{I\PYZus{}d}\PY{p}{)}

\PY{k}{def} \PY{n+nf}{pmos\PYZus{}iv\PYZus{}sweep}\PY{p}{(}\PY{n}{V\PYZus{}sg}\PY{p}{,} \PY{n}{V\PYZus{}sd}\PY{p}{,} \PY{n}{W}\PY{p}{,} \PY{n}{L}\PY{p}{,} \PY{n}{lmda}\PY{p}{)}\PY{p}{:}
    \PY{n}{u\PYZus{}p} \PY{o}{=} \PY{l+m+mi}{100}                 \PY{c+c1}{\PYZsh{} electron mobility (device parameter)}
    \PY{n}{e\PYZus{}ox} \PY{o}{=} \PY{l+m+mf}{3.9}\PY{o}{*}\PY{l+m+mf}{8.854e\PYZhy{}12}\PY{o}{/}\PY{l+m+mi}{100}\PY{p}{;} \PY{c+c1}{\PYZsh{} relative permittivity}
    \PY{n}{t\PYZus{}ox} \PY{o}{=} \PY{l+m+mf}{9e\PYZhy{}9}\PY{o}{*}\PY{l+m+mi}{100}\PY{p}{;}          \PY{c+c1}{\PYZsh{} oxide thickness}
    \PY{n}{C\PYZus{}ox} \PY{o}{=} \PY{n}{e\PYZus{}ox}\PY{o}{/}\PY{n}{t\PYZus{}ox}          \PY{c+c1}{\PYZsh{} oxide capacitance}
    \PY{n}{V\PYZus{}thp} \PY{o}{=} \PY{o}{\PYZhy{}}\PY{l+m+mf}{0.8}                \PY{c+c1}{\PYZsh{} threshold voltage (device parameter)}
    \PY{n}{V\PYZus{}ov} \PY{o}{=} \PY{n}{V\PYZus{}sg} \PY{o}{\PYZhy{}} \PY{n}{np}\PY{o}{.}\PY{n}{abs}\PY{p}{(}\PY{n}{V\PYZus{}thp}\PY{p}{)}
    \PY{n}{Ldp} \PY{o}{=} \PY{l+m+mf}{0.09e\PYZhy{}6}
    \PY{n}{Leff} \PY{o}{=} \PY{n}{L} \PY{o}{\PYZhy{}} \PY{l+m+mi}{2}\PY{o}{*}\PY{n}{Ldp}
    
    \PY{n}{I\PYZus{}d} \PY{o}{=} \PY{p}{[}\PY{p}{]}
    
    \PY{k}{for} \PY{n}{i} \PY{o+ow}{in} \PY{n+nb}{range}\PY{p}{(}\PY{n+nb}{len}\PY{p}{(}\PY{n}{V\PYZus{}sd}\PY{p}{)}\PY{p}{)}\PY{p}{:}
        \PY{n}{I\PYZus{}d}\PY{o}{.}\PY{n}{append}\PY{p}{(}\PY{n}{np}\PY{o}{.}\PY{n}{piecewise}\PY{p}{(}\PY{n}{V\PYZus{}sd}\PY{p}{[}\PY{n}{i}\PY{p}{]}\PY{p}{,} \PY{p}{[}\PY{n}{V\PYZus{}sd}\PY{p}{[}\PY{n}{i}\PY{p}{]} \PY{o}{\PYZlt{}} \PY{n}{V\PYZus{}ov}\PY{p}{,} \PY{n}{V\PYZus{}sd}\PY{p}{[}\PY{n}{i}\PY{p}{]} \PY{o}{\PYZgt{}}\PY{o}{=} \PY{n}{V\PYZus{}ov}\PY{p}{]}\PY{p}{,}
                       \PY{p}{[}\PY{n}{u\PYZus{}p}\PY{o}{*}\PY{n}{C\PYZus{}ox}\PY{o}{*}\PY{p}{(}\PY{n}{W}\PY{o}{/}\PY{n}{Leff}\PY{p}{)}\PY{o}{*}\PY{p}{(}\PY{n}{V\PYZus{}sg} \PY{o}{\PYZhy{}} \PY{n}{np}\PY{o}{.}\PY{n}{abs}\PY{p}{(}\PY{n}{V\PYZus{}thp}\PY{p}{)} \PY{o}{\PYZhy{}} \PY{n}{V\PYZus{}sd}\PY{p}{[}\PY{n}{i}\PY{p}{]}\PY{o}{/}\PY{l+m+mi}{2}\PY{p}{)}\PY{o}{*}\PY{n}{V\PYZus{}sd}\PY{p}{[}\PY{n}{i}\PY{p}{]}\PY{o}{*}\PY{p}{(}\PY{l+m+mi}{1}\PY{o}{+}\PY{n}{lmda}\PY{o}{*}\PY{n}{V\PYZus{}sd}\PY{p}{[}\PY{n}{i}\PY{p}{]}\PY{p}{)} \PY{p}{,} 
                        \PY{l+m+mf}{0.5}\PY{o}{*}\PY{n}{u\PYZus{}p}\PY{o}{*}\PY{n}{C\PYZus{}ox}\PY{o}{*}\PY{p}{(}\PY{n}{W}\PY{o}{/}\PY{n}{Leff}\PY{p}{)}\PY{o}{*}\PY{p}{(}\PY{n}{V\PYZus{}sg} \PY{o}{\PYZhy{}} \PY{n}{np}\PY{o}{.}\PY{n}{abs}\PY{p}{(}\PY{n}{V\PYZus{}thp}\PY{p}{)}\PY{p}{)}\PY{o}{*}\PY{o}{*}\PY{l+m+mi}{2}\PY{o}{*}\PY{p}{(}\PY{l+m+mi}{1}\PY{o}{+}\PY{n}{lmda}\PY{o}{*}\PY{n}{V\PYZus{}sd}\PY{p}{[}\PY{n}{i}\PY{p}{]}\PY{p}{)}\PY{p}{]}\PY{p}{)}\PY{p}{)} 
    
    \PY{k}{return} \PY{n}{np}\PY{o}{.}\PY{n}{array}\PY{p}{(}\PY{n}{I\PYZus{}d}\PY{p}{)}

\PY{k}{def} \PY{n+nf}{nmos\PYZus{}iv\PYZus{}sat}\PY{p}{(}\PY{n}{V\PYZus{}gs}\PY{p}{,} \PY{n}{V\PYZus{}ds}\PY{p}{,} \PY{n}{W}\PY{p}{,} \PY{n}{L}\PY{p}{,} \PY{n}{lmda}\PY{p}{)}\PY{p}{:}
    \PY{n}{u\PYZus{}n} \PY{o}{=} \PY{l+m+mi}{350}                 \PY{c+c1}{\PYZsh{} electron mobility (device parameter)}
    \PY{n}{e\PYZus{}ox} \PY{o}{=} \PY{l+m+mf}{3.9}\PY{o}{*}\PY{l+m+mf}{8.854e\PYZhy{}12}\PY{o}{/}\PY{l+m+mi}{100}\PY{p}{;} \PY{c+c1}{\PYZsh{} relative permittivity}
    \PY{n}{t\PYZus{}ox} \PY{o}{=} \PY{l+m+mf}{9e\PYZhy{}9}\PY{o}{*}\PY{l+m+mi}{100}\PY{p}{;}          \PY{c+c1}{\PYZsh{} oxide thickness}
    \PY{n}{C\PYZus{}ox} \PY{o}{=} \PY{n}{e\PYZus{}ox}\PY{o}{/}\PY{n}{t\PYZus{}ox}          \PY{c+c1}{\PYZsh{} oxide capacitance}
    \PY{n}{V\PYZus{}thn} \PY{o}{=} \PY{l+m+mf}{0.7}                \PY{c+c1}{\PYZsh{} threshold voltage (device parameter)}
    \PY{n}{V\PYZus{}ov} \PY{o}{=} \PY{n}{V\PYZus{}gs} \PY{o}{\PYZhy{}} \PY{n}{V\PYZus{}thn}
    \PY{n}{Ldn} \PY{o}{=} \PY{l+m+mf}{0.08e\PYZhy{}6}
    \PY{n}{Leff} \PY{o}{=} \PY{n}{L} \PY{o}{\PYZhy{}} \PY{l+m+mi}{2}\PY{o}{*}\PY{n}{Ldn}
    
    \PY{n}{I\PYZus{}d} \PY{o}{=} \PY{l+m+mf}{0.5}\PY{o}{*}\PY{n}{u\PYZus{}n}\PY{o}{*}\PY{n}{C\PYZus{}ox}\PY{o}{*}\PY{p}{(}\PY{n}{W}\PY{o}{/}\PY{n}{Leff}\PY{p}{)}\PY{o}{*}\PY{p}{(}\PY{n}{V\PYZus{}gs} \PY{o}{\PYZhy{}} \PY{n}{V\PYZus{}thn}\PY{p}{)}\PY{o}{*}\PY{o}{*}\PY{l+m+mi}{2}\PY{o}{*}\PY{p}{(}\PY{l+m+mi}{1}\PY{o}{+}\PY{n}{lmda}\PY{o}{*}\PY{n}{V\PYZus{}ds}\PY{p}{)}
    
    \PY{k}{return} \PY{n}{I\PYZus{}d}

\PY{k}{def} \PY{n+nf}{nmos\PYZus{}diff\PYZus{}pair}\PY{p}{(}\PY{n}{V\PYZus{}id}\PY{p}{,} \PY{n}{I\PYZus{}ss}\PY{p}{,} \PY{n}{R\PYZus{}D}\PY{p}{,} \PY{n}{W}\PY{p}{,} \PY{n}{L}\PY{p}{,} \PY{n}{V\PYZus{}dd}\PY{p}{)}\PY{p}{:}
    \PY{n}{u\PYZus{}n} \PY{o}{=} \PY{l+m+mi}{350}                 \PY{c+c1}{\PYZsh{} electron mobility (device parameter)}
    \PY{n}{e\PYZus{}ox} \PY{o}{=} \PY{l+m+mf}{3.9}\PY{o}{*}\PY{l+m+mf}{8.854e\PYZhy{}12}\PY{o}{/}\PY{l+m+mi}{100}\PY{p}{;} \PY{c+c1}{\PYZsh{} relative permittivity}
    \PY{n}{t\PYZus{}ox} \PY{o}{=} \PY{l+m+mf}{9e\PYZhy{}9}\PY{o}{*}\PY{l+m+mi}{100}\PY{p}{;}          \PY{c+c1}{\PYZsh{} oxide thickness}
    \PY{n}{C\PYZus{}ox} \PY{o}{=} \PY{n}{e\PYZus{}ox}\PY{o}{/}\PY{n}{t\PYZus{}ox}          \PY{c+c1}{\PYZsh{} oxide capacitance}
    \PY{n}{V\PYZus{}thn} \PY{o}{=} \PY{l+m+mf}{0.7}                \PY{c+c1}{\PYZsh{} threshold voltage (device parameter)}
    \PY{n}{Ldn} \PY{o}{=} \PY{l+m+mf}{0.08e\PYZhy{}6}
    \PY{n}{Leff} \PY{o}{=} \PY{n}{L} \PY{o}{\PYZhy{}} \PY{l+m+mi}{2}\PY{o}{*}\PY{n}{Ldn}
    
    \PY{n}{I\PYZus{}dp} \PY{o}{=} \PY{n}{I\PYZus{}ss}\PY{o}{/}\PY{l+m+mi}{2} \PY{o}{+} \PY{l+m+mf}{0.25}\PY{o}{*}\PY{n}{u\PYZus{}n}\PY{o}{*}\PY{n}{C\PYZus{}ox}\PY{o}{*}\PY{p}{(}\PY{n}{W}\PY{o}{/}\PY{n}{L}\PY{p}{)}\PY{o}{*}\PY{n}{V\PYZus{}id}\PY{o}{*}\PY{n}{np}\PY{o}{.}\PY{n}{sqrt}\PY{p}{(}\PY{l+m+mi}{4}\PY{o}{*}\PY{n}{I\PYZus{}ss}\PY{o}{/}\PY{p}{(}\PY{n}{u\PYZus{}n}\PY{o}{*}\PY{n}{C\PYZus{}ox}\PY{o}{*}\PY{p}{(}\PY{n}{W}\PY{o}{/}\PY{n}{L}\PY{p}{)}\PY{p}{)} \PY{o}{\PYZhy{}} \PY{n}{V\PYZus{}id}\PY{o}{*}\PY{o}{*}\PY{l+m+mi}{2}\PY{p}{)}
    \PY{n}{I\PYZus{}dm} \PY{o}{=} \PY{n}{I\PYZus{}ss}\PY{o}{/}\PY{l+m+mi}{2} \PY{o}{\PYZhy{}} \PY{l+m+mf}{0.25}\PY{o}{*}\PY{n}{u\PYZus{}n}\PY{o}{*}\PY{n}{C\PYZus{}ox}\PY{o}{*}\PY{p}{(}\PY{n}{W}\PY{o}{/}\PY{n}{L}\PY{p}{)}\PY{o}{*}\PY{n}{V\PYZus{}id}\PY{o}{*}\PY{n}{np}\PY{o}{.}\PY{n}{sqrt}\PY{p}{(}\PY{l+m+mi}{4}\PY{o}{*}\PY{n}{I\PYZus{}ss}\PY{o}{/}\PY{p}{(}\PY{n}{u\PYZus{}n}\PY{o}{*}\PY{n}{C\PYZus{}ox}\PY{o}{*}\PY{p}{(}\PY{n}{W}\PY{o}{/}\PY{n}{L}\PY{p}{)}\PY{p}{)} \PY{o}{\PYZhy{}} \PY{n}{V\PYZus{}id}\PY{o}{*}\PY{o}{*}\PY{l+m+mi}{2}\PY{p}{)}

    \PY{k}{return} \PY{n}{I\PYZus{}dp}\PY{p}{,} \PY{n}{I\PYZus{}dm}
\end{Verbatim}
\end{tcolorbox}

    \hypertarget{lecture-7---gmid-design}{%
\section{Lecture 7 - gm/ID Design}\label{lecture-7---gmid-design}}

    \hypertarget{announcements}{%
\subsection{Announcements}\label{announcements}}

    \begin{itemize}
\tightlist
\item
  Midterm exam available now, due February 21

  \begin{itemize}
  \tightlist
  \item
    180-minute time limit, administered as a Canvas quiz
  \end{itemize}
\item
  Design Project Phase 1 posted, due Sunday March 7

  \begin{itemize}
  \tightlist
  \item
    PDF submission on Canvas
  \end{itemize}
\end{itemize}

    \hypertarget{week-7}{%
\subsection{Week 7}\label{week-7}}

    \hypertarget{overview}{%
\subsection{Overview}\label{overview}}

    \begin{itemize}
\tightlist
\item
  Last time\ldots{}

  \begin{itemize}
  \tightlist
  \item
    MOS capacitance
  \item
    Common source amplifier
  \item
    Miller Effect
  \item
    Zero Value Time Constant (ZVTC) analysis
  \item
    Gain-bandwidth product
  \end{itemize}
\item
  Today\ldots{}

  \begin{itemize}
  \tightlist
  \item
    CMOS amplifier design
  \item
    Subthreshold MOS operation
  \item
    \(g_m/I_D\) design methodology
  \end{itemize}
\end{itemize}

    \hypertarget{amplifier-design-considerations}{%
\subsection{Amplifier design
considerations}\label{amplifier-design-considerations}}

    \begin{itemize}
\tightlist
\item
  Analog design involves compromises between

  \begin{itemize}
  \tightlist
  \item
    Speed (circuit bandwidth)
  \item
    Noise (SNR)
  \item
    Power dissipation (supply current)
  \item
    Size (silicon area)
  \end{itemize}
\item
  Design typically begins with a target for one or more of these metrics
\item
  Sometimes, one or more metrics may constitute a degree of freedom
  (e.g.~no noise spec)
\end{itemize}

    \hypertarget{transistor-sizing-methodology}{%
\subsection{Transistor sizing
methodology}\label{transistor-sizing-methodology}}

    \begin{itemize}
\tightlist
\item
  Other limitations/constraints apply, including supply voltage, input
  voltage range, and output swing
\item
  These requirements need to be met while satisfying the core
  specifications (i.e.~power and bandwidth)
\item
  Every transistor should be sized with these targets in mind (i.e.~no
  random selection of \(W/L\) values)
\item
  Transistor sizes should ideally be determined with little to no
  ``tweaking'' in Cadence
\end{itemize}

    \hypertarget{amplifier-design-example}{%
\subsection{Amplifier design example}\label{amplifier-design-example}}

    \begin{figure}
\centering
\includegraphics{5_transistor_OTA.png}
\caption{5\_transistor\_OTA.png}
\end{figure}

    \begin{itemize}
\tightlist
\item
  Specifications:
\end{itemize}

\begin{equation}
A(j0) \approx -g_{m1,2}r_{o2}||r_{o4}  
\end{equation}

\begin{equation}
\omega_{3dB} = \dfrac{1}{r_{o2}||r_{o4} C_L}
\end{equation}

\begin{equation}
GBW = |A(j0)|\cdot \omega_{3dB} \approx \dfrac{g_{m1,2}}{C_L}
\end{equation}

\begin{itemize}
\tightlist
\item
  Amplifier bandwidth/power tradeoff:
\end{itemize}

\begin{equation}
GBW = \dfrac{I_{SS}}{V_{OV1,2}\cdot C_L} = \dfrac{P_{diss}/V_{DD}}{V_{OV1,2}\cdot C_L}
\end{equation}

    \begin{itemize}
\tightlist
\item
  There is a tradeoff between power and bandwidth due to \(I_{SS}\) and
  \(C_L\) limitations
\item
  If both are specified/constrained, design will be a process of
  optimization (e.g.~maximize bandwidth under a given power constraint,
  or minimize power while meeting bandwidth spec)
\end{itemize}

    \hypertarget{amplifier-design-example}{%
\subsection{Amplifier design example}\label{amplifier-design-example}}

    \begin{figure}
\centering
\includegraphics{5_transistor_OTA.png}
\caption{5\_transistor\_OTA.png}
\end{figure}

    \begin{itemize}
\tightlist
\item
  Specifications:
\end{itemize}

\begin{equation}
GBW \approx \dfrac{g_{m1,2}}{C_L} \:\:\:\:\:\:\:\:\: P_{diss} =I_{SS}\cdot V_{DD}
\end{equation}

\begin{itemize}
\tightlist
\item
  Design:
\end{itemize}

\begin{align}
g_{m1,2} &\approx GBW \cdot C_L = \dfrac{I_{SS}}{V_{OV1,2}} \\
\end{align}

\begin{equation}
\left(\dfrac{W}{L}\right)_{1,2} = \dfrac{I_{SS}}{\mu_n C_{ox}V_{OV1,2}^2}
\end{equation}

    \begin{itemize}
\tightlist
\item
  \(V_{DD}\), \(C_L\), and \(P_{diss}\) are set by either specifications
  or technology limitations
\item
  \(M_{1,2}\) should be sized based on the required overdrive voltage
\item
  Selection of \(I_{SS}\) and sizing of \(M_1\), \(M_2\) follows
  directly from specifications
\end{itemize}

    \hypertarget{long-channel-model-limitations}{%
\subsection{Long-channel model
limitations}\label{long-channel-model-limitations}}

    \begin{itemize}
\tightlist
\item
  Unfortunately, modern ``short-channel'' devices do not obey
  long-channel equations
\item
  Starting with these equations typically leads to an iterative,
  \emph{simulate-and-repeat} process, which is counter to our goal of
  ``no tweaking''
\item
  We would like to

  \begin{itemize}
  \tightlist
  \item
    Follow a reasonably simple design approach requiring minimal hand
    analysis, and
  \item
    Obtain accurate results with little to no iteration (i.e.~right the
    first time)
  \end{itemize}
\item
  It turns out that this is largely possibly through an approach called
  \(g_m/I_D\) design
\end{itemize}

    \hypertarget{amplifier-design-revisited}{%
\subsection{Amplifier design,
revisited}\label{amplifier-design-revisited}}

    \begin{figure}
\centering
\includegraphics{5_transistor_OTA.png}
\caption{5\_transistor\_OTA.png}
\end{figure}

    \begin{itemize}
\tightlist
\item
  Specifications:
\end{itemize}

\begin{equation}
GBW \approx \dfrac{g_{m1,2}}{C_L} \:\:\:\:\:\:\:\:\: P_{diss} =I_{SS}\cdot V_{DD}
\end{equation}

\begin{itemize}
\tightlist
\item
  Required value of \(g_m/I_D\):
\end{itemize}

\begin{align}
\dfrac{g_m}{I_D} &\approx \dfrac{2\cdot GBW \cdot C_L}{I_{SS}} = \dfrac{2}{V_{OV}}
\end{align}

    \begin{itemize}
\tightlist
\item
  Viewing specifications in terms of transconductance efficiency
  (\(g_m/I_D\)) provides a model- and operating region-agnostic approach
  to design and sizing
\item
  However, \(g_m/I_D\) depends on process technology and operating
  region, making the simple square-law MOS model inaccurate in many
  cases
\item
  From the above expression for \(g_m/I_D\), it seems like we can just
  continue decreasing \(V_{OV}\) to obtain inifinitely higher
  transconductance efficiency values!
\end{itemize}

    \hypertarget{subthreshold-mos-operation}{%
\subsection{Subthreshold MOS
operation}\label{subthreshold-mos-operation}}

    \begin{figure}
\centering
\includegraphics{MOS_subthreshold.png}
\caption{MOS\_subthreshold.png}
\end{figure}

    \begin{itemize}
\tightlist
\item
  Subthreshold drain current
\end{itemize}

\begin{equation}
I_D = I_S \exp^{\dfrac{V_{GS}}{\zeta V_T}}
\end{equation}

\begin{itemize}
\tightlist
\item
  where
\end{itemize}

\begin{equation}
\zeta = \dfrac{C_{GC} + C_{CB}}{C_{GC}}
\end{equation}

    \begin{itemize}
\tightlist
\item
  As \(V_{OV}\) decreases, \(V_{GS}\) becomes equal to/less than
  \(V_{th}\)
\item
  For \(V_{GS}\) values less than \(V_{th}\), the bulk region under the
  gate has a high concentration of majority carriers (holes for NMOS)
\item
  Increasing/decreasing the gate potential in this regime lowers/raises
  the potential barrier for diffusion of majority carriers between the
  S/B regions (this is similar to BJT operation)
\item
  \(\zeta\) is a process parameter, typically in the range 1.5 to 2
\end{itemize}

    \hypertarget{subthreshold-transconductance}{%
\subsection{Subthreshold
transconductance}\label{subthreshold-transconductance}}

    \begin{itemize}
\tightlist
\item
  For subthreshold operation (also called weak inversion),
  transconductance is given as
\end{itemize}

\begin{equation}
g_m =\dfrac{\partial I_D}{\partial V_{GS}} = \dfrac{\partial}{\partial V_{GS}} I_S \exp^{\dfrac{V_{GS}}{\zeta V_T}} = \dfrac{I_S \exp^{\dfrac{V_{GS}}{\zeta V_T}}}{\zeta V_T} = \boxed{ \dfrac{I_D}{\zeta V_T}}
\end{equation}

\begin{itemize}
\tightlist
\item
  This an analogous expression to the transconductance of a \(BJT\),
  with the exception of the \(\zeta\) term
\item
  As in the square-law model, transconductance varies linearly with
  \(I_D\)
\item
  Both drain current and transconductance are in fact continuous
  functions of \(V_{GS}\), and the models merely serve as approximations
\item
  It should be noted that the accuracy of both square law and
  exponential models is poor for \(V_{GS} \approx V_{th}\)
\end{itemize}

    \hypertarget{transconductance-efficiency-gmid}{%
\subsection{Transconductance efficiency
(gm/ID)}\label{transconductance-efficiency-gmid}}

    \begin{itemize}
\tightlist
\item
  In saturation, the transconductance is given by
\end{itemize}

\begin{equation}
g_m = \dfrac{2I_D}{V_{OV}}
\end{equation}

\begin{itemize}
\tightlist
\item
  From this we can define the transconductance efficiency, given by
\end{itemize}

\begin{equation}
\dfrac{g_m}{I_D} = \dfrac{2}{V_{OV}}
\end{equation}

\begin{itemize}
\tightlist
\item
  For a MOS transistor in subthreshold, this becomes
\end{itemize}

\begin{equation}
\dfrac{g_m}{I_D} = \dfrac{1}{\zeta V_T}
\end{equation}

\begin{itemize}
\tightlist
\item
  At room temperature (\(300K\)) for \(\zeta \approx 1.5\), this is
  limited to approximately
\end{itemize}

\begin{equation}
\dfrac{g_m}{I_D} \approx 26 S/A
\end{equation}

    \hypertarget{gmid-design-example}{%
\subsection{gm/ID design example}\label{gmid-design-example}}

    \begin{figure}
\centering
\includegraphics{5_transistor_OTA.png}
\caption{5\_transistor\_OTA.png}
\end{figure}

    \begin{itemize}
\tightlist
\item
  Specifications:
\end{itemize}

\begin{equation}
V_{DD} = 2.5V \:\:\:\:\:\:\:\: C_L = 10pF \\
P_{diss} =  320\mu W \:\:\:\:\:\:\:\: GBW = 10MHz 
\end{equation}

\begin{itemize}
\tightlist
\item
  Design calculations:
\end{itemize}

\begin{equation}
I_{D1,2} = \dfrac{P_{diss}}{2V_{DD}} = 64 \mu A
\end{equation}

\begin{equation}
g_{m1,2} = 2 \pi \cdot 10MHz \cdot 10 pF = 630 \mu S
\end{equation}

\begin{equation}
\dfrac{g_{m}}{I_{D}} = \dfrac{630 \mu S}{64 \mu A} \approx 10V^{-1}
\end{equation}

    \hypertarget{gmid-current-density-chart}{%
\subsection{gm/ID current density
chart}\label{gmid-current-density-chart}}

    \begin{figure}
\centering
\includegraphics{gm_over_ID_vs_ID_over_W.png}
\caption{gm\_over\_ID\_vs\_ID\_over\_W.png}
\end{figure}

    \hypertarget{input-pair-sizing}{%
\subsection{Input pair sizing}\label{input-pair-sizing}}

    \begin{itemize}
\tightlist
\item
  The allowable current for \(M_{1,2}\) is based on the power
  dissipation spec:
\end{itemize}

\begin{equation}
I_{D1,2} = \dfrac{P_{diss}}{2V_{DD}} = 64 \mu A
\end{equation}

\begin{itemize}
\tightlist
\item
  The current density chart allows us to directly select \(W\) based on
  the target \(g_m/I_D\) and the target drain current \(I_{D1,2}\)
\end{itemize}

\begin{equation}
W_{1,2} = \dfrac{I_{D1,2}}{I_D/W} = \dfrac{64\mu A}{2.6 \mu A/\mu m} = 25.6 \mu m
\end{equation}

\begin{itemize}
\tightlist
\item
  We can use the same approach to size bias devices in terms of
  \(V_{OV}\) with the corresponding charts (e.g.~\(V_{OV}\) vs
  \(I_D/W\))
\end{itemize}

    \begin{figure}
\centering
\includegraphics{OTA_schematic_cadence.png}
\caption{OTA\_schematic\_cadence.png}
\end{figure}

    \begin{figure}
\centering
\includegraphics{OTA_operating_point.png}
\caption{OTA\_operating\_point.png}
\end{figure}

    \hypertarget{simulation-results}{%
\subsection{Simulation results}\label{simulation-results}}

    \begin{figure}
\centering
\includegraphics{OTA_bode_plot.png}
\caption{OTA\_bode\_plot.png}
\end{figure}

    \hypertarget{gmid-design-results}{%
\subsection{gm/ID design results}\label{gmid-design-results}}

    \begin{itemize}
\tightlist
\item
  Our design is essentially right on target, with little to no need for
  tweaking
\item
  This result was achieved without using any of the long-channel
  equations
\item
  Much less model uncertainty, since the model complexity is ``built
  in'' to the \(g_m/I_D\) chart generated by simulation
\item
  \(g_m/I_D\) design enables the designer to think in terms of
  transconductance efficiency and current density, improving intuition
  relative to an equation-based approach
\end{itemize}

    \hypertarget{general-design-flow}{%
\subsection{General design flow}\label{general-design-flow}}

    \begin{itemize}
\tightlist
\item
  Determine the required value(s) of \(g_m\), \(g_m r_o\), or \(V_{OV}\)
  from design objectives (e.g.~bandwidth, gain, output swing)
\item
  Select channel length \(L\)

  \begin{itemize}
  \tightlist
  \item
    Short channel \(\rightarrow\) high \(f_T\) (high speed)
  \item
    Long channel \(\rightarrow\) high intrinsic gain, good matching
  \end{itemize}
\item
  Determine required value of \(g_m/I_D\)

  \begin{itemize}
  \tightlist
  \item
    Large \(g_m/I_D\) \(\rightarrow\) low power or large signal swing
    (low \(V_{OV}\))
  \item
    Small \(g_m/I_D\) \(\rightarrow\) high speed, better linearity
  \end{itemize}
\item
  Determine \(I_D\) from \(g_m\) and \(g_m/I_D\)
\item
  Determine \(W\) from current density (\(I_D/W\)) chart
\end{itemize}

    \hypertarget{summary}{%
\subsection{Summary}\label{summary}}

    \begin{itemize}
\tightlist
\item
  Analog CMOS design can (and should) be a systematic process

  \begin{itemize}
  \tightlist
  \item
    Specifications \(\rightarrow\) circuit architecture \(\rightarrow\)
    device currents/sizes
  \end{itemize}
\item
  The long-channel MOS model is inadequate for accurately predicting
  device behavior in short-channel CMOS technologies or at low current
  densities (subthreshold)
\item
  \(g_m/I_D\) design offers a unified approach that captures complex
  device model behavior while still allowing the use of simple
  expressions for making design decisions
\end{itemize}


    % Add a bibliography block to the postdoc
    
    
    
\end{document}
