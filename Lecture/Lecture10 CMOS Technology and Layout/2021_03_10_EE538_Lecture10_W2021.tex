\documentclass[11pt]{article}

    \usepackage[breakable]{tcolorbox}
    \usepackage{parskip} % Stop auto-indenting (to mimic markdown behaviour)
    
    \usepackage{iftex}
    \ifPDFTeX
    	\usepackage[T1]{fontenc}
    	\usepackage{mathpazo}
    \else
    	\usepackage{fontspec}
    \fi

    % Basic figure setup, for now with no caption control since it's done
    % automatically by Pandoc (which extracts ![](path) syntax from Markdown).
    \usepackage{graphicx}
    % Maintain compatibility with old templates. Remove in nbconvert 6.0
    \let\Oldincludegraphics\includegraphics
    % Ensure that by default, figures have no caption (until we provide a
    % proper Figure object with a Caption API and a way to capture that
    % in the conversion process - todo).
    \usepackage{caption}
    \DeclareCaptionFormat{nocaption}{}
    \captionsetup{format=nocaption,aboveskip=0pt,belowskip=0pt}

    \usepackage[Export]{adjustbox} % Used to constrain images to a maximum size
    \adjustboxset{max size={0.9\linewidth}{0.9\paperheight}}
    \usepackage{float}
    \floatplacement{figure}{H} % forces figures to be placed at the correct location
    \usepackage{xcolor} % Allow colors to be defined
    \usepackage{enumerate} % Needed for markdown enumerations to work
    \usepackage{geometry} % Used to adjust the document margins
    \usepackage{amsmath} % Equations
    \usepackage{amssymb} % Equations
    \usepackage{textcomp} % defines textquotesingle
    % Hack from http://tex.stackexchange.com/a/47451/13684:
    \AtBeginDocument{%
        \def\PYZsq{\textquotesingle}% Upright quotes in Pygmentized code
    }
    \usepackage{upquote} % Upright quotes for verbatim code
    \usepackage{eurosym} % defines \euro
    \usepackage[mathletters]{ucs} % Extended unicode (utf-8) support
    \usepackage{fancyvrb} % verbatim replacement that allows latex
    \usepackage{grffile} % extends the file name processing of package graphics 
                         % to support a larger range
    \makeatletter % fix for grffile with XeLaTeX
    \def\Gread@@xetex#1{%
      \IfFileExists{"\Gin@base".bb}%
      {\Gread@eps{\Gin@base.bb}}%
      {\Gread@@xetex@aux#1}%
    }
    \makeatother

    % The hyperref package gives us a pdf with properly built
    % internal navigation ('pdf bookmarks' for the table of contents,
    % internal cross-reference links, web links for URLs, etc.)
    \usepackage{hyperref}
    % The default LaTeX title has an obnoxious amount of whitespace. By default,
    % titling removes some of it. It also provides customization options.
    \usepackage{titling}
    \usepackage{longtable} % longtable support required by pandoc >1.10
    \usepackage{booktabs}  % table support for pandoc > 1.12.2
    \usepackage[inline]{enumitem} % IRkernel/repr support (it uses the enumerate* environment)
    \usepackage[normalem]{ulem} % ulem is needed to support strikethroughs (\sout)
                                % normalem makes italics be italics, not underlines
    \usepackage{mathrsfs}
    

    
    % Colors for the hyperref package
    \definecolor{urlcolor}{rgb}{0,.145,.698}
    \definecolor{linkcolor}{rgb}{.71,0.21,0.01}
    \definecolor{citecolor}{rgb}{.12,.54,.11}

    % ANSI colors
    \definecolor{ansi-black}{HTML}{3E424D}
    \definecolor{ansi-black-intense}{HTML}{282C36}
    \definecolor{ansi-red}{HTML}{E75C58}
    \definecolor{ansi-red-intense}{HTML}{B22B31}
    \definecolor{ansi-green}{HTML}{00A250}
    \definecolor{ansi-green-intense}{HTML}{007427}
    \definecolor{ansi-yellow}{HTML}{DDB62B}
    \definecolor{ansi-yellow-intense}{HTML}{B27D12}
    \definecolor{ansi-blue}{HTML}{208FFB}
    \definecolor{ansi-blue-intense}{HTML}{0065CA}
    \definecolor{ansi-magenta}{HTML}{D160C4}
    \definecolor{ansi-magenta-intense}{HTML}{A03196}
    \definecolor{ansi-cyan}{HTML}{60C6C8}
    \definecolor{ansi-cyan-intense}{HTML}{258F8F}
    \definecolor{ansi-white}{HTML}{C5C1B4}
    \definecolor{ansi-white-intense}{HTML}{A1A6B2}
    \definecolor{ansi-default-inverse-fg}{HTML}{FFFFFF}
    \definecolor{ansi-default-inverse-bg}{HTML}{000000}

    % commands and environments needed by pandoc snippets
    % extracted from the output of `pandoc -s`
    \providecommand{\tightlist}{%
      \setlength{\itemsep}{0pt}\setlength{\parskip}{0pt}}
    \DefineVerbatimEnvironment{Highlighting}{Verbatim}{commandchars=\\\{\}}
    % Add ',fontsize=\small' for more characters per line
    \newenvironment{Shaded}{}{}
    \newcommand{\KeywordTok}[1]{\textcolor[rgb]{0.00,0.44,0.13}{\textbf{{#1}}}}
    \newcommand{\DataTypeTok}[1]{\textcolor[rgb]{0.56,0.13,0.00}{{#1}}}
    \newcommand{\DecValTok}[1]{\textcolor[rgb]{0.25,0.63,0.44}{{#1}}}
    \newcommand{\BaseNTok}[1]{\textcolor[rgb]{0.25,0.63,0.44}{{#1}}}
    \newcommand{\FloatTok}[1]{\textcolor[rgb]{0.25,0.63,0.44}{{#1}}}
    \newcommand{\CharTok}[1]{\textcolor[rgb]{0.25,0.44,0.63}{{#1}}}
    \newcommand{\StringTok}[1]{\textcolor[rgb]{0.25,0.44,0.63}{{#1}}}
    \newcommand{\CommentTok}[1]{\textcolor[rgb]{0.38,0.63,0.69}{\textit{{#1}}}}
    \newcommand{\OtherTok}[1]{\textcolor[rgb]{0.00,0.44,0.13}{{#1}}}
    \newcommand{\AlertTok}[1]{\textcolor[rgb]{1.00,0.00,0.00}{\textbf{{#1}}}}
    \newcommand{\FunctionTok}[1]{\textcolor[rgb]{0.02,0.16,0.49}{{#1}}}
    \newcommand{\RegionMarkerTok}[1]{{#1}}
    \newcommand{\ErrorTok}[1]{\textcolor[rgb]{1.00,0.00,0.00}{\textbf{{#1}}}}
    \newcommand{\NormalTok}[1]{{#1}}
    
    % Additional commands for more recent versions of Pandoc
    \newcommand{\ConstantTok}[1]{\textcolor[rgb]{0.53,0.00,0.00}{{#1}}}
    \newcommand{\SpecialCharTok}[1]{\textcolor[rgb]{0.25,0.44,0.63}{{#1}}}
    \newcommand{\VerbatimStringTok}[1]{\textcolor[rgb]{0.25,0.44,0.63}{{#1}}}
    \newcommand{\SpecialStringTok}[1]{\textcolor[rgb]{0.73,0.40,0.53}{{#1}}}
    \newcommand{\ImportTok}[1]{{#1}}
    \newcommand{\DocumentationTok}[1]{\textcolor[rgb]{0.73,0.13,0.13}{\textit{{#1}}}}
    \newcommand{\AnnotationTok}[1]{\textcolor[rgb]{0.38,0.63,0.69}{\textbf{\textit{{#1}}}}}
    \newcommand{\CommentVarTok}[1]{\textcolor[rgb]{0.38,0.63,0.69}{\textbf{\textit{{#1}}}}}
    \newcommand{\VariableTok}[1]{\textcolor[rgb]{0.10,0.09,0.49}{{#1}}}
    \newcommand{\ControlFlowTok}[1]{\textcolor[rgb]{0.00,0.44,0.13}{\textbf{{#1}}}}
    \newcommand{\OperatorTok}[1]{\textcolor[rgb]{0.40,0.40,0.40}{{#1}}}
    \newcommand{\BuiltInTok}[1]{{#1}}
    \newcommand{\ExtensionTok}[1]{{#1}}
    \newcommand{\PreprocessorTok}[1]{\textcolor[rgb]{0.74,0.48,0.00}{{#1}}}
    \newcommand{\AttributeTok}[1]{\textcolor[rgb]{0.49,0.56,0.16}{{#1}}}
    \newcommand{\InformationTok}[1]{\textcolor[rgb]{0.38,0.63,0.69}{\textbf{\textit{{#1}}}}}
    \newcommand{\WarningTok}[1]{\textcolor[rgb]{0.38,0.63,0.69}{\textbf{\textit{{#1}}}}}
    
    
    % Define a nice break command that doesn't care if a line doesn't already
    % exist.
    \def\br{\hspace*{\fill} \\* }
    % Math Jax compatibility definitions
    \def\gt{>}
    \def\lt{<}
    \let\Oldtex\TeX
    \let\Oldlatex\LaTeX
    \renewcommand{\TeX}{\textrm{\Oldtex}}
    \renewcommand{\LaTeX}{\textrm{\Oldlatex}}
    % Document parameters
    % Document title
    \title{2021\_03\_10\_EE538\_Lecture10\_W2021}
    
    
    
    
    
% Pygments definitions
\makeatletter
\def\PY@reset{\let\PY@it=\relax \let\PY@bf=\relax%
    \let\PY@ul=\relax \let\PY@tc=\relax%
    \let\PY@bc=\relax \let\PY@ff=\relax}
\def\PY@tok#1{\csname PY@tok@#1\endcsname}
\def\PY@toks#1+{\ifx\relax#1\empty\else%
    \PY@tok{#1}\expandafter\PY@toks\fi}
\def\PY@do#1{\PY@bc{\PY@tc{\PY@ul{%
    \PY@it{\PY@bf{\PY@ff{#1}}}}}}}
\def\PY#1#2{\PY@reset\PY@toks#1+\relax+\PY@do{#2}}

\expandafter\def\csname PY@tok@w\endcsname{\def\PY@tc##1{\textcolor[rgb]{0.73,0.73,0.73}{##1}}}
\expandafter\def\csname PY@tok@c\endcsname{\let\PY@it=\textit\def\PY@tc##1{\textcolor[rgb]{0.25,0.50,0.50}{##1}}}
\expandafter\def\csname PY@tok@cp\endcsname{\def\PY@tc##1{\textcolor[rgb]{0.74,0.48,0.00}{##1}}}
\expandafter\def\csname PY@tok@k\endcsname{\let\PY@bf=\textbf\def\PY@tc##1{\textcolor[rgb]{0.00,0.50,0.00}{##1}}}
\expandafter\def\csname PY@tok@kp\endcsname{\def\PY@tc##1{\textcolor[rgb]{0.00,0.50,0.00}{##1}}}
\expandafter\def\csname PY@tok@kt\endcsname{\def\PY@tc##1{\textcolor[rgb]{0.69,0.00,0.25}{##1}}}
\expandafter\def\csname PY@tok@o\endcsname{\def\PY@tc##1{\textcolor[rgb]{0.40,0.40,0.40}{##1}}}
\expandafter\def\csname PY@tok@ow\endcsname{\let\PY@bf=\textbf\def\PY@tc##1{\textcolor[rgb]{0.67,0.13,1.00}{##1}}}
\expandafter\def\csname PY@tok@nb\endcsname{\def\PY@tc##1{\textcolor[rgb]{0.00,0.50,0.00}{##1}}}
\expandafter\def\csname PY@tok@nf\endcsname{\def\PY@tc##1{\textcolor[rgb]{0.00,0.00,1.00}{##1}}}
\expandafter\def\csname PY@tok@nc\endcsname{\let\PY@bf=\textbf\def\PY@tc##1{\textcolor[rgb]{0.00,0.00,1.00}{##1}}}
\expandafter\def\csname PY@tok@nn\endcsname{\let\PY@bf=\textbf\def\PY@tc##1{\textcolor[rgb]{0.00,0.00,1.00}{##1}}}
\expandafter\def\csname PY@tok@ne\endcsname{\let\PY@bf=\textbf\def\PY@tc##1{\textcolor[rgb]{0.82,0.25,0.23}{##1}}}
\expandafter\def\csname PY@tok@nv\endcsname{\def\PY@tc##1{\textcolor[rgb]{0.10,0.09,0.49}{##1}}}
\expandafter\def\csname PY@tok@no\endcsname{\def\PY@tc##1{\textcolor[rgb]{0.53,0.00,0.00}{##1}}}
\expandafter\def\csname PY@tok@nl\endcsname{\def\PY@tc##1{\textcolor[rgb]{0.63,0.63,0.00}{##1}}}
\expandafter\def\csname PY@tok@ni\endcsname{\let\PY@bf=\textbf\def\PY@tc##1{\textcolor[rgb]{0.60,0.60,0.60}{##1}}}
\expandafter\def\csname PY@tok@na\endcsname{\def\PY@tc##1{\textcolor[rgb]{0.49,0.56,0.16}{##1}}}
\expandafter\def\csname PY@tok@nt\endcsname{\let\PY@bf=\textbf\def\PY@tc##1{\textcolor[rgb]{0.00,0.50,0.00}{##1}}}
\expandafter\def\csname PY@tok@nd\endcsname{\def\PY@tc##1{\textcolor[rgb]{0.67,0.13,1.00}{##1}}}
\expandafter\def\csname PY@tok@s\endcsname{\def\PY@tc##1{\textcolor[rgb]{0.73,0.13,0.13}{##1}}}
\expandafter\def\csname PY@tok@sd\endcsname{\let\PY@it=\textit\def\PY@tc##1{\textcolor[rgb]{0.73,0.13,0.13}{##1}}}
\expandafter\def\csname PY@tok@si\endcsname{\let\PY@bf=\textbf\def\PY@tc##1{\textcolor[rgb]{0.73,0.40,0.53}{##1}}}
\expandafter\def\csname PY@tok@se\endcsname{\let\PY@bf=\textbf\def\PY@tc##1{\textcolor[rgb]{0.73,0.40,0.13}{##1}}}
\expandafter\def\csname PY@tok@sr\endcsname{\def\PY@tc##1{\textcolor[rgb]{0.73,0.40,0.53}{##1}}}
\expandafter\def\csname PY@tok@ss\endcsname{\def\PY@tc##1{\textcolor[rgb]{0.10,0.09,0.49}{##1}}}
\expandafter\def\csname PY@tok@sx\endcsname{\def\PY@tc##1{\textcolor[rgb]{0.00,0.50,0.00}{##1}}}
\expandafter\def\csname PY@tok@m\endcsname{\def\PY@tc##1{\textcolor[rgb]{0.40,0.40,0.40}{##1}}}
\expandafter\def\csname PY@tok@gh\endcsname{\let\PY@bf=\textbf\def\PY@tc##1{\textcolor[rgb]{0.00,0.00,0.50}{##1}}}
\expandafter\def\csname PY@tok@gu\endcsname{\let\PY@bf=\textbf\def\PY@tc##1{\textcolor[rgb]{0.50,0.00,0.50}{##1}}}
\expandafter\def\csname PY@tok@gd\endcsname{\def\PY@tc##1{\textcolor[rgb]{0.63,0.00,0.00}{##1}}}
\expandafter\def\csname PY@tok@gi\endcsname{\def\PY@tc##1{\textcolor[rgb]{0.00,0.63,0.00}{##1}}}
\expandafter\def\csname PY@tok@gr\endcsname{\def\PY@tc##1{\textcolor[rgb]{1.00,0.00,0.00}{##1}}}
\expandafter\def\csname PY@tok@ge\endcsname{\let\PY@it=\textit}
\expandafter\def\csname PY@tok@gs\endcsname{\let\PY@bf=\textbf}
\expandafter\def\csname PY@tok@gp\endcsname{\let\PY@bf=\textbf\def\PY@tc##1{\textcolor[rgb]{0.00,0.00,0.50}{##1}}}
\expandafter\def\csname PY@tok@go\endcsname{\def\PY@tc##1{\textcolor[rgb]{0.53,0.53,0.53}{##1}}}
\expandafter\def\csname PY@tok@gt\endcsname{\def\PY@tc##1{\textcolor[rgb]{0.00,0.27,0.87}{##1}}}
\expandafter\def\csname PY@tok@err\endcsname{\def\PY@bc##1{\setlength{\fboxsep}{0pt}\fcolorbox[rgb]{1.00,0.00,0.00}{1,1,1}{\strut ##1}}}
\expandafter\def\csname PY@tok@kc\endcsname{\let\PY@bf=\textbf\def\PY@tc##1{\textcolor[rgb]{0.00,0.50,0.00}{##1}}}
\expandafter\def\csname PY@tok@kd\endcsname{\let\PY@bf=\textbf\def\PY@tc##1{\textcolor[rgb]{0.00,0.50,0.00}{##1}}}
\expandafter\def\csname PY@tok@kn\endcsname{\let\PY@bf=\textbf\def\PY@tc##1{\textcolor[rgb]{0.00,0.50,0.00}{##1}}}
\expandafter\def\csname PY@tok@kr\endcsname{\let\PY@bf=\textbf\def\PY@tc##1{\textcolor[rgb]{0.00,0.50,0.00}{##1}}}
\expandafter\def\csname PY@tok@bp\endcsname{\def\PY@tc##1{\textcolor[rgb]{0.00,0.50,0.00}{##1}}}
\expandafter\def\csname PY@tok@fm\endcsname{\def\PY@tc##1{\textcolor[rgb]{0.00,0.00,1.00}{##1}}}
\expandafter\def\csname PY@tok@vc\endcsname{\def\PY@tc##1{\textcolor[rgb]{0.10,0.09,0.49}{##1}}}
\expandafter\def\csname PY@tok@vg\endcsname{\def\PY@tc##1{\textcolor[rgb]{0.10,0.09,0.49}{##1}}}
\expandafter\def\csname PY@tok@vi\endcsname{\def\PY@tc##1{\textcolor[rgb]{0.10,0.09,0.49}{##1}}}
\expandafter\def\csname PY@tok@vm\endcsname{\def\PY@tc##1{\textcolor[rgb]{0.10,0.09,0.49}{##1}}}
\expandafter\def\csname PY@tok@sa\endcsname{\def\PY@tc##1{\textcolor[rgb]{0.73,0.13,0.13}{##1}}}
\expandafter\def\csname PY@tok@sb\endcsname{\def\PY@tc##1{\textcolor[rgb]{0.73,0.13,0.13}{##1}}}
\expandafter\def\csname PY@tok@sc\endcsname{\def\PY@tc##1{\textcolor[rgb]{0.73,0.13,0.13}{##1}}}
\expandafter\def\csname PY@tok@dl\endcsname{\def\PY@tc##1{\textcolor[rgb]{0.73,0.13,0.13}{##1}}}
\expandafter\def\csname PY@tok@s2\endcsname{\def\PY@tc##1{\textcolor[rgb]{0.73,0.13,0.13}{##1}}}
\expandafter\def\csname PY@tok@sh\endcsname{\def\PY@tc##1{\textcolor[rgb]{0.73,0.13,0.13}{##1}}}
\expandafter\def\csname PY@tok@s1\endcsname{\def\PY@tc##1{\textcolor[rgb]{0.73,0.13,0.13}{##1}}}
\expandafter\def\csname PY@tok@mb\endcsname{\def\PY@tc##1{\textcolor[rgb]{0.40,0.40,0.40}{##1}}}
\expandafter\def\csname PY@tok@mf\endcsname{\def\PY@tc##1{\textcolor[rgb]{0.40,0.40,0.40}{##1}}}
\expandafter\def\csname PY@tok@mh\endcsname{\def\PY@tc##1{\textcolor[rgb]{0.40,0.40,0.40}{##1}}}
\expandafter\def\csname PY@tok@mi\endcsname{\def\PY@tc##1{\textcolor[rgb]{0.40,0.40,0.40}{##1}}}
\expandafter\def\csname PY@tok@il\endcsname{\def\PY@tc##1{\textcolor[rgb]{0.40,0.40,0.40}{##1}}}
\expandafter\def\csname PY@tok@mo\endcsname{\def\PY@tc##1{\textcolor[rgb]{0.40,0.40,0.40}{##1}}}
\expandafter\def\csname PY@tok@ch\endcsname{\let\PY@it=\textit\def\PY@tc##1{\textcolor[rgb]{0.25,0.50,0.50}{##1}}}
\expandafter\def\csname PY@tok@cm\endcsname{\let\PY@it=\textit\def\PY@tc##1{\textcolor[rgb]{0.25,0.50,0.50}{##1}}}
\expandafter\def\csname PY@tok@cpf\endcsname{\let\PY@it=\textit\def\PY@tc##1{\textcolor[rgb]{0.25,0.50,0.50}{##1}}}
\expandafter\def\csname PY@tok@c1\endcsname{\let\PY@it=\textit\def\PY@tc##1{\textcolor[rgb]{0.25,0.50,0.50}{##1}}}
\expandafter\def\csname PY@tok@cs\endcsname{\let\PY@it=\textit\def\PY@tc##1{\textcolor[rgb]{0.25,0.50,0.50}{##1}}}

\def\PYZbs{\char`\\}
\def\PYZus{\char`\_}
\def\PYZob{\char`\{}
\def\PYZcb{\char`\}}
\def\PYZca{\char`\^}
\def\PYZam{\char`\&}
\def\PYZlt{\char`\<}
\def\PYZgt{\char`\>}
\def\PYZsh{\char`\#}
\def\PYZpc{\char`\%}
\def\PYZdl{\char`\$}
\def\PYZhy{\char`\-}
\def\PYZsq{\char`\'}
\def\PYZdq{\char`\"}
\def\PYZti{\char`\~}
% for compatibility with earlier versions
\def\PYZat{@}
\def\PYZlb{[}
\def\PYZrb{]}
\makeatother


    % For linebreaks inside Verbatim environment from package fancyvrb. 
    \makeatletter
        \newbox\Wrappedcontinuationbox 
        \newbox\Wrappedvisiblespacebox 
        \newcommand*\Wrappedvisiblespace {\textcolor{red}{\textvisiblespace}} 
        \newcommand*\Wrappedcontinuationsymbol {\textcolor{red}{\llap{\tiny$\m@th\hookrightarrow$}}} 
        \newcommand*\Wrappedcontinuationindent {3ex } 
        \newcommand*\Wrappedafterbreak {\kern\Wrappedcontinuationindent\copy\Wrappedcontinuationbox} 
        % Take advantage of the already applied Pygments mark-up to insert 
        % potential linebreaks for TeX processing. 
        %        {, <, #, %, $, ' and ": go to next line. 
        %        _, }, ^, &, >, - and ~: stay at end of broken line. 
        % Use of \textquotesingle for straight quote. 
        \newcommand*\Wrappedbreaksatspecials {% 
            \def\PYGZus{\discretionary{\char`\_}{\Wrappedafterbreak}{\char`\_}}% 
            \def\PYGZob{\discretionary{}{\Wrappedafterbreak\char`\{}{\char`\{}}% 
            \def\PYGZcb{\discretionary{\char`\}}{\Wrappedafterbreak}{\char`\}}}% 
            \def\PYGZca{\discretionary{\char`\^}{\Wrappedafterbreak}{\char`\^}}% 
            \def\PYGZam{\discretionary{\char`\&}{\Wrappedafterbreak}{\char`\&}}% 
            \def\PYGZlt{\discretionary{}{\Wrappedafterbreak\char`\<}{\char`\<}}% 
            \def\PYGZgt{\discretionary{\char`\>}{\Wrappedafterbreak}{\char`\>}}% 
            \def\PYGZsh{\discretionary{}{\Wrappedafterbreak\char`\#}{\char`\#}}% 
            \def\PYGZpc{\discretionary{}{\Wrappedafterbreak\char`\%}{\char`\%}}% 
            \def\PYGZdl{\discretionary{}{\Wrappedafterbreak\char`\$}{\char`\$}}% 
            \def\PYGZhy{\discretionary{\char`\-}{\Wrappedafterbreak}{\char`\-}}% 
            \def\PYGZsq{\discretionary{}{\Wrappedafterbreak\textquotesingle}{\textquotesingle}}% 
            \def\PYGZdq{\discretionary{}{\Wrappedafterbreak\char`\"}{\char`\"}}% 
            \def\PYGZti{\discretionary{\char`\~}{\Wrappedafterbreak}{\char`\~}}% 
        } 
        % Some characters . , ; ? ! / are not pygmentized. 
        % This macro makes them "active" and they will insert potential linebreaks 
        \newcommand*\Wrappedbreaksatpunct {% 
            \lccode`\~`\.\lowercase{\def~}{\discretionary{\hbox{\char`\.}}{\Wrappedafterbreak}{\hbox{\char`\.}}}% 
            \lccode`\~`\,\lowercase{\def~}{\discretionary{\hbox{\char`\,}}{\Wrappedafterbreak}{\hbox{\char`\,}}}% 
            \lccode`\~`\;\lowercase{\def~}{\discretionary{\hbox{\char`\;}}{\Wrappedafterbreak}{\hbox{\char`\;}}}% 
            \lccode`\~`\:\lowercase{\def~}{\discretionary{\hbox{\char`\:}}{\Wrappedafterbreak}{\hbox{\char`\:}}}% 
            \lccode`\~`\?\lowercase{\def~}{\discretionary{\hbox{\char`\?}}{\Wrappedafterbreak}{\hbox{\char`\?}}}% 
            \lccode`\~`\!\lowercase{\def~}{\discretionary{\hbox{\char`\!}}{\Wrappedafterbreak}{\hbox{\char`\!}}}% 
            \lccode`\~`\/\lowercase{\def~}{\discretionary{\hbox{\char`\/}}{\Wrappedafterbreak}{\hbox{\char`\/}}}% 
            \catcode`\.\active
            \catcode`\,\active 
            \catcode`\;\active
            \catcode`\:\active
            \catcode`\?\active
            \catcode`\!\active
            \catcode`\/\active 
            \lccode`\~`\~ 	
        }
    \makeatother

    \let\OriginalVerbatim=\Verbatim
    \makeatletter
    \renewcommand{\Verbatim}[1][1]{%
        %\parskip\z@skip
        \sbox\Wrappedcontinuationbox {\Wrappedcontinuationsymbol}%
        \sbox\Wrappedvisiblespacebox {\FV@SetupFont\Wrappedvisiblespace}%
        \def\FancyVerbFormatLine ##1{\hsize\linewidth
            \vtop{\raggedright\hyphenpenalty\z@\exhyphenpenalty\z@
                \doublehyphendemerits\z@\finalhyphendemerits\z@
                \strut ##1\strut}%
        }%
        % If the linebreak is at a space, the latter will be displayed as visible
        % space at end of first line, and a continuation symbol starts next line.
        % Stretch/shrink are however usually zero for typewriter font.
        \def\FV@Space {%
            \nobreak\hskip\z@ plus\fontdimen3\font minus\fontdimen4\font
            \discretionary{\copy\Wrappedvisiblespacebox}{\Wrappedafterbreak}
            {\kern\fontdimen2\font}%
        }%
        
        % Allow breaks at special characters using \PYG... macros.
        \Wrappedbreaksatspecials
        % Breaks at punctuation characters . , ; ? ! and / need catcode=\active 	
        \OriginalVerbatim[#1,codes*=\Wrappedbreaksatpunct]%
    }
    \makeatother

    % Exact colors from NB
    \definecolor{incolor}{HTML}{303F9F}
    \definecolor{outcolor}{HTML}{D84315}
    \definecolor{cellborder}{HTML}{CFCFCF}
    \definecolor{cellbackground}{HTML}{F7F7F7}
    
    % prompt
    \makeatletter
    \newcommand{\boxspacing}{\kern\kvtcb@left@rule\kern\kvtcb@boxsep}
    \makeatother
    \newcommand{\prompt}[4]{
        \ttfamily\llap{{\color{#2}[#3]:\hspace{3pt}#4}}\vspace{-\baselineskip}
    }
    

    
    % Prevent overflowing lines due to hard-to-break entities
    \sloppy 
    % Setup hyperref package
    \hypersetup{
      breaklinks=true,  % so long urls are correctly broken across lines
      colorlinks=true,
      urlcolor=urlcolor,
      linkcolor=linkcolor,
      citecolor=citecolor,
      }
    % Slightly bigger margins than the latex defaults
    
    \geometry{verbose,tmargin=1in,bmargin=1in,lmargin=1in,rmargin=1in}
    
    

\begin{document}
    
    \maketitle
    
    

    
    \hypertarget{ee-538-analog-integrated-circuit-design}{%
\section{EE 538: Analog Integrated Circuit
Design}\label{ee-538-analog-integrated-circuit-design}}

\hypertarget{winter-2021}{%
\subsection{Winter 2021}\label{winter-2021}}

\hypertarget{instructor-jason-silver}{%
\subsection{Instructor: Jason Silver}\label{instructor-jason-silver}}

    \hypertarget{python-packagesmodules}{%
\subsection{Python packages/modules}\label{python-packagesmodules}}

    \begin{tcolorbox}[breakable, size=fbox, boxrule=1pt, pad at break*=1mm,colback=cellbackground, colframe=cellborder]
\prompt{In}{incolor}{2}{\boxspacing}
\begin{Verbatim}[commandchars=\\\{\}]
\PY{k+kn}{import} \PY{n+nn}{matplotlib} \PY{k}{as} \PY{n+nn}{mpl}
\PY{k+kn}{from} \PY{n+nn}{matplotlib} \PY{k+kn}{import} \PY{n}{pyplot} \PY{k}{as} \PY{n}{plt}
\PY{k+kn}{import} \PY{n+nn}{numpy} \PY{k}{as} \PY{n+nn}{np}
\PY{k+kn}{from} \PY{n+nn}{scipy} \PY{k+kn}{import} \PY{n}{signal}
\PY{c+c1}{\PYZsh{}\PYZpc{}matplotlib notebook}

\PY{n}{mpl}\PY{o}{.}\PY{n}{rcParams}\PY{p}{[}\PY{l+s+s1}{\PYZsq{}}\PY{l+s+s1}{font.size}\PY{l+s+s1}{\PYZsq{}}\PY{p}{]} \PY{o}{=} \PY{l+m+mi}{12}
\PY{n}{mpl}\PY{o}{.}\PY{n}{rcParams}\PY{p}{[}\PY{l+s+s1}{\PYZsq{}}\PY{l+s+s1}{legend.fontsize}\PY{l+s+s1}{\PYZsq{}}\PY{p}{]} \PY{o}{=} \PY{l+s+s1}{\PYZsq{}}\PY{l+s+s1}{large}\PY{l+s+s1}{\PYZsq{}}

\PY{k}{def} \PY{n+nf}{plot\PYZus{}xy}\PY{p}{(}\PY{n}{x}\PY{p}{,} \PY{n}{y}\PY{p}{,} \PY{n}{xlabel}\PY{p}{,} \PY{n}{ylabel}\PY{p}{)}\PY{p}{:}
    \PY{n}{fig}\PY{p}{,} \PY{n}{ax} \PY{o}{=} \PY{n}{plt}\PY{o}{.}\PY{n}{subplots}\PY{p}{(}\PY{n}{figsize}\PY{o}{=}\PY{p}{(}\PY{l+m+mf}{10.0}\PY{p}{,} \PY{l+m+mf}{7.5}\PY{p}{)}\PY{p}{)}\PY{p}{;}
    \PY{n}{ax}\PY{o}{.}\PY{n}{plot}\PY{p}{(}\PY{n}{x}\PY{p}{,} \PY{n}{y}\PY{p}{,} \PY{l+s+s1}{\PYZsq{}}\PY{l+s+s1}{b}\PY{l+s+s1}{\PYZsq{}}\PY{p}{)}\PY{p}{;}
    \PY{n}{ax}\PY{o}{.}\PY{n}{grid}\PY{p}{(}\PY{p}{)}\PY{p}{;}
    \PY{n}{ax}\PY{o}{.}\PY{n}{set\PYZus{}xlabel}\PY{p}{(}\PY{n}{xlabel}\PY{p}{)}\PY{p}{;}
    \PY{n}{ax}\PY{o}{.}\PY{n}{set\PYZus{}ylabel}\PY{p}{(}\PY{n}{ylabel}\PY{p}{)}\PY{p}{;}
    
\PY{k}{def} \PY{n+nf}{plot\PYZus{}xy2}\PY{p}{(}\PY{n}{x1}\PY{p}{,} \PY{n}{y1}\PY{p}{,} \PY{n}{x1label}\PY{p}{,} \PY{n}{y1label}\PY{p}{,} \PY{n}{x2}\PY{p}{,} \PY{n}{y2}\PY{p}{,} \PY{n}{x2label}\PY{p}{,} \PY{n}{y2label}\PY{p}{)}\PY{p}{:}
    \PY{n}{fig}\PY{p}{,} \PY{n}{ax} \PY{o}{=} \PY{n}{plt}\PY{o}{.}\PY{n}{subplots}\PY{p}{(}\PY{l+m+mi}{2}\PY{p}{,} \PY{n}{figsize} \PY{o}{=} \PY{p}{(}\PY{l+m+mf}{10.0}\PY{p}{,} \PY{l+m+mf}{7.5}\PY{p}{)}\PY{p}{)}\PY{p}{;}
    \PY{n}{ax}\PY{p}{[}\PY{l+m+mi}{0}\PY{p}{]}\PY{o}{.}\PY{n}{plot}\PY{p}{(}\PY{n}{x1}\PY{p}{,} \PY{n}{y1}\PY{p}{,} \PY{l+s+s1}{\PYZsq{}}\PY{l+s+s1}{b}\PY{l+s+s1}{\PYZsq{}}\PY{p}{)}\PY{p}{;}
    \PY{n}{ax}\PY{p}{[}\PY{l+m+mi}{0}\PY{p}{]}\PY{o}{.}\PY{n}{set\PYZus{}ylabel}\PY{p}{(}\PY{n}{y1label}\PY{p}{)}
    \PY{n}{ax}\PY{p}{[}\PY{l+m+mi}{0}\PY{p}{]}\PY{o}{.}\PY{n}{grid}\PY{p}{(}\PY{p}{)}
    
    \PY{n}{ax}\PY{p}{[}\PY{l+m+mi}{1}\PY{p}{]}\PY{o}{.}\PY{n}{plot}\PY{p}{(}\PY{n}{x2}\PY{p}{,} \PY{n}{y2}\PY{p}{,} \PY{l+s+s1}{\PYZsq{}}\PY{l+s+s1}{b}\PY{l+s+s1}{\PYZsq{}}\PY{p}{)}\PY{p}{;}
    \PY{n}{ax}\PY{p}{[}\PY{l+m+mi}{1}\PY{p}{]}\PY{o}{.}\PY{n}{set\PYZus{}xlabel}\PY{p}{(}\PY{n}{x1label}\PY{p}{)}
    \PY{n}{ax}\PY{p}{[}\PY{l+m+mi}{1}\PY{p}{]}\PY{o}{.}\PY{n}{set\PYZus{}xlabel}\PY{p}{(}\PY{n}{x2label}\PY{p}{)}\PY{p}{;}
    \PY{n}{ax}\PY{p}{[}\PY{l+m+mi}{1}\PY{p}{]}\PY{o}{.}\PY{n}{set\PYZus{}ylabel}\PY{p}{(}\PY{n}{y2label}\PY{p}{)}\PY{p}{;}
    \PY{n}{ax}\PY{p}{[}\PY{l+m+mi}{1}\PY{p}{]}\PY{o}{.}\PY{n}{grid}\PY{p}{(}\PY{p}{)}\PY{p}{;}
    
    \PY{n}{fig}\PY{o}{.}\PY{n}{align\PYZus{}ylabels}\PY{p}{(}\PY{n}{ax}\PY{p}{[}\PY{p}{:}\PY{p}{]}\PY{p}{)}
    
\PY{k}{def} \PY{n+nf}{plot\PYZus{}x2y}\PY{p}{(}\PY{n}{x}\PY{p}{,} \PY{n}{y1}\PY{p}{,} \PY{n}{y2}\PY{p}{,} \PY{n}{xlabel}\PY{p}{,} \PY{n}{ylabel}\PY{p}{,} \PY{n}{y1label}\PY{p}{,} \PY{n}{y2label}\PY{p}{)}\PY{p}{:}
        
    \PY{n}{fig}\PY{p}{,} \PY{n}{ax} \PY{o}{=} \PY{n}{plt}\PY{o}{.}\PY{n}{subplots}\PY{p}{(}\PY{n}{figsize}\PY{o}{=}\PY{p}{(}\PY{l+m+mf}{10.0}\PY{p}{,} \PY{l+m+mf}{7.5}\PY{p}{)}\PY{p}{)}\PY{p}{;}
    \PY{n}{ax}\PY{o}{.}\PY{n}{plot}\PY{p}{(}\PY{n}{x}\PY{p}{,} \PY{n}{y1}\PY{p}{,} \PY{l+s+s1}{\PYZsq{}}\PY{l+s+s1}{b}\PY{l+s+s1}{\PYZsq{}}\PY{p}{)}
    \PY{n}{ax}\PY{o}{.}\PY{n}{plot}\PY{p}{(}\PY{n}{x}\PY{p}{,} \PY{n}{y2}\PY{p}{,} \PY{l+s+s1}{\PYZsq{}}\PY{l+s+s1}{r}\PY{l+s+s1}{\PYZsq{}}\PY{p}{)}
    \PY{n}{ax}\PY{o}{.}\PY{n}{legend}\PY{p}{(} \PY{p}{[}\PY{n}{y1label}\PY{p}{,}\PY{n}{y2label}\PY{p}{]} \PY{p}{,}\PY{n}{loc}\PY{o}{=}\PY{l+s+s1}{\PYZsq{}}\PY{l+s+s1}{upper center}\PY{l+s+s1}{\PYZsq{}}\PY{p}{,} \PY{n}{ncol}\PY{o}{=}\PY{l+m+mi}{5}\PY{p}{,} \PY{n}{fancybox}\PY{o}{=}\PY{k+kc}{True}\PY{p}{,} 
           \PY{n}{shadow}\PY{o}{=}\PY{k+kc}{True}\PY{p}{,} \PY{n}{bbox\PYZus{}to\PYZus{}anchor}\PY{o}{=}\PY{p}{(}\PY{l+m+mf}{0.5}\PY{p}{,}\PY{l+m+mf}{1.1}\PY{p}{)}\PY{p}{)}  
    \PY{n}{ax}\PY{o}{.}\PY{n}{grid}\PY{p}{(}\PY{p}{)}
    \PY{n}{ax}\PY{o}{.}\PY{n}{set\PYZus{}xlabel}\PY{p}{(}\PY{n}{xlabel}\PY{p}{)}
    \PY{n}{ax}\PY{o}{.}\PY{n}{set\PYZus{}ylabel}\PY{p}{(}\PY{n}{ylabel}\PY{p}{)}
    
\PY{k}{def} \PY{n+nf}{plot\PYZus{}xy3}\PY{p}{(}\PY{n}{x}\PY{p}{,} \PY{n}{y1}\PY{p}{,} \PY{n}{y2}\PY{p}{,} \PY{n}{y3}\PY{p}{,} \PY{n}{xlabel}\PY{p}{,} \PY{n}{y1label}\PY{p}{,} \PY{n}{y2label}\PY{p}{,} \PY{n}{y3label}\PY{p}{)}\PY{p}{:}
    \PY{n}{fig}\PY{p}{,} \PY{n}{ax} \PY{o}{=} \PY{n}{plt}\PY{o}{.}\PY{n}{subplots}\PY{p}{(}\PY{l+m+mi}{3}\PY{p}{,} \PY{n}{figsize}\PY{o}{=}\PY{p}{(}\PY{l+m+mf}{10.0}\PY{p}{,}\PY{l+m+mf}{7.5}\PY{p}{)}\PY{p}{)}
    
    \PY{n}{ax}\PY{p}{[}\PY{l+m+mi}{0}\PY{p}{]}\PY{o}{.}\PY{n}{plot}\PY{p}{(}\PY{n}{x}\PY{p}{,} \PY{n}{y1}\PY{p}{)}
    \PY{n}{ax}\PY{p}{[}\PY{l+m+mi}{0}\PY{p}{]}\PY{o}{.}\PY{n}{set\PYZus{}ylabel}\PY{p}{(}\PY{n}{y1label}\PY{p}{)}
    \PY{n}{ax}\PY{p}{[}\PY{l+m+mi}{0}\PY{p}{]}\PY{o}{.}\PY{n}{grid}\PY{p}{(}\PY{p}{)}
    
    \PY{n}{ax}\PY{p}{[}\PY{l+m+mi}{1}\PY{p}{]}\PY{o}{.}\PY{n}{plot}\PY{p}{(}\PY{n}{x}\PY{p}{,} \PY{n}{y2}\PY{p}{)}
    \PY{n}{ax}\PY{p}{[}\PY{l+m+mi}{1}\PY{p}{]}\PY{o}{.}\PY{n}{set\PYZus{}ylabel}\PY{p}{(}\PY{n}{y2label}\PY{p}{)}
    \PY{n}{ax}\PY{p}{[}\PY{l+m+mi}{1}\PY{p}{]}\PY{o}{.}\PY{n}{grid}\PY{p}{(}\PY{p}{)}
    
    \PY{n}{ax}\PY{p}{[}\PY{l+m+mi}{2}\PY{p}{]}\PY{o}{.}\PY{n}{plot}\PY{p}{(}\PY{n}{x}\PY{p}{,} \PY{n}{y3}\PY{p}{)}  
    \PY{n}{ax}\PY{p}{[}\PY{l+m+mi}{2}\PY{p}{]}\PY{o}{.}\PY{n}{set\PYZus{}ylabel}\PY{p}{(}\PY{n}{y3label}\PY{p}{)}
    \PY{n}{ax}\PY{p}{[}\PY{l+m+mi}{2}\PY{p}{]}\PY{o}{.}\PY{n}{set\PYZus{}xlabel}\PY{p}{(}\PY{n}{xlabel}\PY{p}{)}
    \PY{n}{ax}\PY{p}{[}\PY{l+m+mi}{2}\PY{p}{]}\PY{o}{.}\PY{n}{grid}\PY{p}{(}\PY{p}{)}
    
\PY{k}{def} \PY{n+nf}{plot\PYZus{}xlogy}\PY{p}{(}\PY{n}{x}\PY{p}{,} \PY{n}{y}\PY{p}{,} \PY{n}{xlabel}\PY{p}{,} \PY{n}{ylabel}\PY{p}{)}\PY{p}{:}
    \PY{n}{fig}\PY{p}{,} \PY{n}{ax} \PY{o}{=} \PY{n}{plt}\PY{o}{.}\PY{n}{subplots}\PY{p}{(}\PY{n}{figsize}\PY{o}{=}\PY{p}{(}\PY{l+m+mf}{10.0}\PY{p}{,} \PY{l+m+mf}{7.5}\PY{p}{)}\PY{p}{)}\PY{p}{;}
    \PY{n}{ax}\PY{o}{.}\PY{n}{semilogy}\PY{p}{(}\PY{n}{x}\PY{p}{,} \PY{n}{y}\PY{p}{,} \PY{l+s+s1}{\PYZsq{}}\PY{l+s+s1}{b}\PY{l+s+s1}{\PYZsq{}}\PY{p}{)}\PY{p}{;}
    \PY{n}{ax}\PY{o}{.}\PY{n}{grid}\PY{p}{(}\PY{p}{)}\PY{p}{;}
    \PY{n}{ax}\PY{o}{.}\PY{n}{set\PYZus{}xlabel}\PY{p}{(}\PY{n}{xlabel}\PY{p}{)}\PY{p}{;}
    \PY{n}{ax}\PY{o}{.}\PY{n}{set\PYZus{}ylabel}\PY{p}{(}\PY{n}{ylabel}\PY{p}{)}\PY{p}{;}

\PY{k}{def} \PY{n+nf}{plot\PYZus{}logxy2}\PY{p}{(}\PY{n}{x1}\PY{p}{,} \PY{n}{y1}\PY{p}{,} \PY{n}{x2}\PY{p}{,} \PY{n}{y2}\PY{p}{,} \PY{n}{x1label}\PY{p}{,} \PY{n}{y1label}\PY{p}{,} \PY{n}{x2label}\PY{p}{,} \PY{n}{y2label}\PY{p}{)}\PY{p}{:}
    \PY{n}{fig}\PY{p}{,} \PY{n}{ax} \PY{o}{=} \PY{n}{plt}\PY{o}{.}\PY{n}{subplots}\PY{p}{(}\PY{l+m+mi}{2}\PY{p}{,} \PY{n}{figsize} \PY{o}{=} \PY{p}{(}\PY{l+m+mf}{10.0}\PY{p}{,} \PY{l+m+mf}{7.5}\PY{p}{)}\PY{p}{)}\PY{p}{;}
    \PY{n}{ax}\PY{p}{[}\PY{l+m+mi}{0}\PY{p}{]}\PY{o}{.}\PY{n}{semilogx}\PY{p}{(}\PY{n}{x1}\PY{p}{,} \PY{n}{y1}\PY{p}{,} \PY{l+s+s1}{\PYZsq{}}\PY{l+s+s1}{b}\PY{l+s+s1}{\PYZsq{}}\PY{p}{)}\PY{p}{;}
    \PY{n}{ax}\PY{p}{[}\PY{l+m+mi}{0}\PY{p}{]}\PY{o}{.}\PY{n}{set\PYZus{}ylabel}\PY{p}{(}\PY{n}{y1label}\PY{p}{)}
    \PY{n}{ax}\PY{p}{[}\PY{l+m+mi}{0}\PY{p}{]}\PY{o}{.}\PY{n}{grid}\PY{p}{(}\PY{p}{)}
    
    \PY{n}{ax}\PY{p}{[}\PY{l+m+mi}{1}\PY{p}{]}\PY{o}{.}\PY{n}{semilogx}\PY{p}{(}\PY{n}{x2}\PY{p}{,} \PY{n}{y2}\PY{p}{,} \PY{l+s+s1}{\PYZsq{}}\PY{l+s+s1}{b}\PY{l+s+s1}{\PYZsq{}}\PY{p}{)}\PY{p}{;}
    \PY{n}{ax}\PY{p}{[}\PY{l+m+mi}{1}\PY{p}{]}\PY{o}{.}\PY{n}{set\PYZus{}xlabel}\PY{p}{(}\PY{n}{x1label}\PY{p}{)}
    \PY{n}{ax}\PY{p}{[}\PY{l+m+mi}{1}\PY{p}{]}\PY{o}{.}\PY{n}{set\PYZus{}xlabel}\PY{p}{(}\PY{n}{x2label}\PY{p}{)}\PY{p}{;}
    \PY{n}{ax}\PY{p}{[}\PY{l+m+mi}{1}\PY{p}{]}\PY{o}{.}\PY{n}{set\PYZus{}ylabel}\PY{p}{(}\PY{n}{y2label}\PY{p}{)}\PY{p}{;}
    \PY{n}{ax}\PY{p}{[}\PY{l+m+mi}{1}\PY{p}{]}\PY{o}{.}\PY{n}{grid}\PY{p}{(}\PY{p}{)}\PY{p}{;}
    
    \PY{n}{fig}\PY{o}{.}\PY{n}{align\PYZus{}ylabels}\PY{p}{(}\PY{n}{ax}\PY{p}{[}\PY{p}{:}\PY{p}{]}\PY{p}{)}    
    
\PY{k}{def} \PY{n+nf}{nmos\PYZus{}iv\PYZus{}sweep}\PY{p}{(}\PY{n}{V\PYZus{}gs}\PY{p}{,} \PY{n}{V\PYZus{}ds}\PY{p}{,} \PY{n}{W}\PY{p}{,} \PY{n}{L}\PY{p}{,} \PY{n}{lmda}\PY{p}{)}\PY{p}{:}
    \PY{n}{u\PYZus{}n} \PY{o}{=} \PY{l+m+mi}{350}                 \PY{c+c1}{\PYZsh{} electron mobility (device parameter)}
    \PY{n}{e\PYZus{}ox} \PY{o}{=} \PY{l+m+mf}{3.9}\PY{o}{*}\PY{l+m+mf}{8.854e\PYZhy{}12}\PY{o}{/}\PY{l+m+mi}{100}\PY{p}{;} \PY{c+c1}{\PYZsh{} relative permittivity}
    \PY{n}{t\PYZus{}ox} \PY{o}{=} \PY{l+m+mf}{9e\PYZhy{}9}\PY{o}{*}\PY{l+m+mi}{100}\PY{p}{;}          \PY{c+c1}{\PYZsh{} oxide thickness}
    \PY{n}{C\PYZus{}ox} \PY{o}{=} \PY{n}{e\PYZus{}ox}\PY{o}{/}\PY{n}{t\PYZus{}ox}          \PY{c+c1}{\PYZsh{} oxide capacitance}
    \PY{n}{V\PYZus{}thn} \PY{o}{=} \PY{l+m+mf}{0.7}                \PY{c+c1}{\PYZsh{} threshold voltage (device parameter)}
    \PY{n}{V\PYZus{}ov} \PY{o}{=} \PY{n}{V\PYZus{}gs} \PY{o}{\PYZhy{}} \PY{n}{V\PYZus{}thn}
    \PY{n}{Ldn} \PY{o}{=} \PY{l+m+mf}{0.08e\PYZhy{}6}
    \PY{n}{Leff} \PY{o}{=} \PY{n}{L} \PY{o}{\PYZhy{}} \PY{l+m+mi}{2}\PY{o}{*}\PY{n}{Ldn}
    
    \PY{n}{I\PYZus{}d} \PY{o}{=} \PY{p}{[}\PY{p}{]}
    
    \PY{k}{for} \PY{n}{i} \PY{o+ow}{in} \PY{n+nb}{range}\PY{p}{(}\PY{n+nb}{len}\PY{p}{(}\PY{n}{V\PYZus{}ds}\PY{p}{)}\PY{p}{)}\PY{p}{:}
        \PY{n}{I\PYZus{}d}\PY{o}{.}\PY{n}{append}\PY{p}{(}\PY{n}{np}\PY{o}{.}\PY{n}{piecewise}\PY{p}{(}\PY{n}{V\PYZus{}ds}\PY{p}{[}\PY{n}{i}\PY{p}{]}\PY{p}{,} \PY{p}{[}\PY{n}{V\PYZus{}ds}\PY{p}{[}\PY{n}{i}\PY{p}{]} \PY{o}{\PYZlt{}} \PY{n}{V\PYZus{}ov}\PY{p}{,} \PY{n}{V\PYZus{}ds}\PY{p}{[}\PY{n}{i}\PY{p}{]} \PY{o}{\PYZgt{}}\PY{o}{=} \PY{n}{V\PYZus{}ov}\PY{p}{]}\PY{p}{,}
                       \PY{p}{[}\PY{n}{u\PYZus{}n}\PY{o}{*}\PY{n}{C\PYZus{}ox}\PY{o}{*}\PY{p}{(}\PY{n}{W}\PY{o}{/}\PY{n}{Leff}\PY{p}{)}\PY{o}{*}\PY{p}{(}\PY{n}{V\PYZus{}gs} \PY{o}{\PYZhy{}} \PY{n}{V\PYZus{}thn} \PY{o}{\PYZhy{}} \PY{n}{V\PYZus{}ds}\PY{p}{[}\PY{n}{i}\PY{p}{]}\PY{o}{/}\PY{l+m+mi}{2}\PY{p}{)}\PY{o}{*}\PY{n}{V\PYZus{}ds}\PY{p}{[}\PY{n}{i}\PY{p}{]}\PY{o}{*}\PY{p}{(}\PY{l+m+mi}{1}\PY{o}{+}\PY{n}{lmda}\PY{o}{*}\PY{n}{V\PYZus{}ds}\PY{p}{[}\PY{n}{i}\PY{p}{]}\PY{p}{)} \PY{p}{,} 
                        \PY{l+m+mf}{0.5}\PY{o}{*}\PY{n}{u\PYZus{}n}\PY{o}{*}\PY{n}{C\PYZus{}ox}\PY{o}{*}\PY{p}{(}\PY{n}{W}\PY{o}{/}\PY{n}{Leff}\PY{p}{)}\PY{o}{*}\PY{p}{(}\PY{n}{V\PYZus{}gs} \PY{o}{\PYZhy{}} \PY{n}{V\PYZus{}thn}\PY{p}{)}\PY{o}{*}\PY{o}{*}\PY{l+m+mi}{2}\PY{o}{*}\PY{p}{(}\PY{l+m+mi}{1}\PY{o}{+}\PY{n}{lmda}\PY{o}{*}\PY{n}{V\PYZus{}ds}\PY{p}{[}\PY{n}{i}\PY{p}{]}\PY{p}{)}\PY{p}{]}\PY{p}{)}\PY{p}{)} 
    
    \PY{k}{return} \PY{n}{np}\PY{o}{.}\PY{n}{array}\PY{p}{(}\PY{n}{I\PYZus{}d}\PY{p}{)}

\PY{k}{def} \PY{n+nf}{pmos\PYZus{}iv\PYZus{}sweep}\PY{p}{(}\PY{n}{V\PYZus{}sg}\PY{p}{,} \PY{n}{V\PYZus{}sd}\PY{p}{,} \PY{n}{W}\PY{p}{,} \PY{n}{L}\PY{p}{,} \PY{n}{lmda}\PY{p}{)}\PY{p}{:}
    \PY{n}{u\PYZus{}p} \PY{o}{=} \PY{l+m+mi}{100}                 \PY{c+c1}{\PYZsh{} electron mobility (device parameter)}
    \PY{n}{e\PYZus{}ox} \PY{o}{=} \PY{l+m+mf}{3.9}\PY{o}{*}\PY{l+m+mf}{8.854e\PYZhy{}12}\PY{o}{/}\PY{l+m+mi}{100}\PY{p}{;} \PY{c+c1}{\PYZsh{} relative permittivity}
    \PY{n}{t\PYZus{}ox} \PY{o}{=} \PY{l+m+mf}{9e\PYZhy{}9}\PY{o}{*}\PY{l+m+mi}{100}\PY{p}{;}          \PY{c+c1}{\PYZsh{} oxide thickness}
    \PY{n}{C\PYZus{}ox} \PY{o}{=} \PY{n}{e\PYZus{}ox}\PY{o}{/}\PY{n}{t\PYZus{}ox}          \PY{c+c1}{\PYZsh{} oxide capacitance}
    \PY{n}{V\PYZus{}thp} \PY{o}{=} \PY{o}{\PYZhy{}}\PY{l+m+mf}{0.8}                \PY{c+c1}{\PYZsh{} threshold voltage (device parameter)}
    \PY{n}{V\PYZus{}ov} \PY{o}{=} \PY{n}{V\PYZus{}sg} \PY{o}{\PYZhy{}} \PY{n}{np}\PY{o}{.}\PY{n}{abs}\PY{p}{(}\PY{n}{V\PYZus{}thp}\PY{p}{)}
    \PY{n}{Ldp} \PY{o}{=} \PY{l+m+mf}{0.09e\PYZhy{}6}
    \PY{n}{Leff} \PY{o}{=} \PY{n}{L} \PY{o}{\PYZhy{}} \PY{l+m+mi}{2}\PY{o}{*}\PY{n}{Ldp}
    
    \PY{n}{I\PYZus{}d} \PY{o}{=} \PY{p}{[}\PY{p}{]}
    
    \PY{k}{for} \PY{n}{i} \PY{o+ow}{in} \PY{n+nb}{range}\PY{p}{(}\PY{n+nb}{len}\PY{p}{(}\PY{n}{V\PYZus{}sd}\PY{p}{)}\PY{p}{)}\PY{p}{:}
        \PY{n}{I\PYZus{}d}\PY{o}{.}\PY{n}{append}\PY{p}{(}\PY{n}{np}\PY{o}{.}\PY{n}{piecewise}\PY{p}{(}\PY{n}{V\PYZus{}sd}\PY{p}{[}\PY{n}{i}\PY{p}{]}\PY{p}{,} \PY{p}{[}\PY{n}{V\PYZus{}sd}\PY{p}{[}\PY{n}{i}\PY{p}{]} \PY{o}{\PYZlt{}} \PY{n}{V\PYZus{}ov}\PY{p}{,} \PY{n}{V\PYZus{}sd}\PY{p}{[}\PY{n}{i}\PY{p}{]} \PY{o}{\PYZgt{}}\PY{o}{=} \PY{n}{V\PYZus{}ov}\PY{p}{]}\PY{p}{,}
                       \PY{p}{[}\PY{n}{u\PYZus{}p}\PY{o}{*}\PY{n}{C\PYZus{}ox}\PY{o}{*}\PY{p}{(}\PY{n}{W}\PY{o}{/}\PY{n}{Leff}\PY{p}{)}\PY{o}{*}\PY{p}{(}\PY{n}{V\PYZus{}sg} \PY{o}{\PYZhy{}} \PY{n}{np}\PY{o}{.}\PY{n}{abs}\PY{p}{(}\PY{n}{V\PYZus{}thp}\PY{p}{)} \PY{o}{\PYZhy{}} \PY{n}{V\PYZus{}sd}\PY{p}{[}\PY{n}{i}\PY{p}{]}\PY{o}{/}\PY{l+m+mi}{2}\PY{p}{)}\PY{o}{*}\PY{n}{V\PYZus{}sd}\PY{p}{[}\PY{n}{i}\PY{p}{]}\PY{o}{*}\PY{p}{(}\PY{l+m+mi}{1}\PY{o}{+}\PY{n}{lmda}\PY{o}{*}\PY{n}{V\PYZus{}sd}\PY{p}{[}\PY{n}{i}\PY{p}{]}\PY{p}{)} \PY{p}{,} 
                        \PY{l+m+mf}{0.5}\PY{o}{*}\PY{n}{u\PYZus{}p}\PY{o}{*}\PY{n}{C\PYZus{}ox}\PY{o}{*}\PY{p}{(}\PY{n}{W}\PY{o}{/}\PY{n}{Leff}\PY{p}{)}\PY{o}{*}\PY{p}{(}\PY{n}{V\PYZus{}sg} \PY{o}{\PYZhy{}} \PY{n}{np}\PY{o}{.}\PY{n}{abs}\PY{p}{(}\PY{n}{V\PYZus{}thp}\PY{p}{)}\PY{p}{)}\PY{o}{*}\PY{o}{*}\PY{l+m+mi}{2}\PY{o}{*}\PY{p}{(}\PY{l+m+mi}{1}\PY{o}{+}\PY{n}{lmda}\PY{o}{*}\PY{n}{V\PYZus{}sd}\PY{p}{[}\PY{n}{i}\PY{p}{]}\PY{p}{)}\PY{p}{]}\PY{p}{)}\PY{p}{)} 
    
    \PY{k}{return} \PY{n}{np}\PY{o}{.}\PY{n}{array}\PY{p}{(}\PY{n}{I\PYZus{}d}\PY{p}{)}

\PY{k}{def} \PY{n+nf}{nmos\PYZus{}iv\PYZus{}sat}\PY{p}{(}\PY{n}{V\PYZus{}gs}\PY{p}{,} \PY{n}{V\PYZus{}ds}\PY{p}{,} \PY{n}{W}\PY{p}{,} \PY{n}{L}\PY{p}{,} \PY{n}{lmda}\PY{p}{)}\PY{p}{:}
    \PY{n}{u\PYZus{}n} \PY{o}{=} \PY{l+m+mi}{350}                 \PY{c+c1}{\PYZsh{} electron mobility (device parameter)}
    \PY{n}{e\PYZus{}ox} \PY{o}{=} \PY{l+m+mf}{3.9}\PY{o}{*}\PY{l+m+mf}{8.854e\PYZhy{}12}\PY{o}{/}\PY{l+m+mi}{100}\PY{p}{;} \PY{c+c1}{\PYZsh{} relative permittivity}
    \PY{n}{t\PYZus{}ox} \PY{o}{=} \PY{l+m+mf}{9e\PYZhy{}9}\PY{o}{*}\PY{l+m+mi}{100}\PY{p}{;}          \PY{c+c1}{\PYZsh{} oxide thickness}
    \PY{n}{C\PYZus{}ox} \PY{o}{=} \PY{n}{e\PYZus{}ox}\PY{o}{/}\PY{n}{t\PYZus{}ox}          \PY{c+c1}{\PYZsh{} oxide capacitance}
    \PY{n}{V\PYZus{}thn} \PY{o}{=} \PY{l+m+mf}{0.7}                \PY{c+c1}{\PYZsh{} threshold voltage (device parameter)}
    \PY{n}{V\PYZus{}ov} \PY{o}{=} \PY{n}{V\PYZus{}gs} \PY{o}{\PYZhy{}} \PY{n}{V\PYZus{}thn}
    \PY{n}{Ldn} \PY{o}{=} \PY{l+m+mf}{0.08e\PYZhy{}6}
    \PY{n}{Leff} \PY{o}{=} \PY{n}{L} \PY{o}{\PYZhy{}} \PY{l+m+mi}{2}\PY{o}{*}\PY{n}{Ldn}
    
    \PY{n}{I\PYZus{}d} \PY{o}{=} \PY{l+m+mf}{0.5}\PY{o}{*}\PY{n}{u\PYZus{}n}\PY{o}{*}\PY{n}{C\PYZus{}ox}\PY{o}{*}\PY{p}{(}\PY{n}{W}\PY{o}{/}\PY{n}{Leff}\PY{p}{)}\PY{o}{*}\PY{p}{(}\PY{n}{V\PYZus{}gs} \PY{o}{\PYZhy{}} \PY{n}{V\PYZus{}thn}\PY{p}{)}\PY{o}{*}\PY{o}{*}\PY{l+m+mi}{2}\PY{o}{*}\PY{p}{(}\PY{l+m+mi}{1}\PY{o}{+}\PY{n}{lmda}\PY{o}{*}\PY{n}{V\PYZus{}ds}\PY{p}{)}
    
    \PY{k}{return} \PY{n}{I\PYZus{}d}

\PY{k}{def} \PY{n+nf}{nmos\PYZus{}diff\PYZus{}pair}\PY{p}{(}\PY{n}{V\PYZus{}id}\PY{p}{,} \PY{n}{I\PYZus{}ss}\PY{p}{,} \PY{n}{R\PYZus{}D}\PY{p}{,} \PY{n}{W}\PY{p}{,} \PY{n}{L}\PY{p}{,} \PY{n}{V\PYZus{}dd}\PY{p}{)}\PY{p}{:}
    \PY{n}{u\PYZus{}n} \PY{o}{=} \PY{l+m+mi}{350}                 \PY{c+c1}{\PYZsh{} electron mobility (device parameter)}
    \PY{n}{e\PYZus{}ox} \PY{o}{=} \PY{l+m+mf}{3.9}\PY{o}{*}\PY{l+m+mf}{8.854e\PYZhy{}12}\PY{o}{/}\PY{l+m+mi}{100}\PY{p}{;} \PY{c+c1}{\PYZsh{} relative permittivity}
    \PY{n}{t\PYZus{}ox} \PY{o}{=} \PY{l+m+mf}{9e\PYZhy{}9}\PY{o}{*}\PY{l+m+mi}{100}\PY{p}{;}          \PY{c+c1}{\PYZsh{} oxide thickness}
    \PY{n}{C\PYZus{}ox} \PY{o}{=} \PY{n}{e\PYZus{}ox}\PY{o}{/}\PY{n}{t\PYZus{}ox}          \PY{c+c1}{\PYZsh{} oxide capacitance}
    \PY{n}{V\PYZus{}thn} \PY{o}{=} \PY{l+m+mf}{0.7}                \PY{c+c1}{\PYZsh{} threshold voltage (device parameter)}
    \PY{n}{Ldn} \PY{o}{=} \PY{l+m+mf}{0.08e\PYZhy{}6}
    \PY{n}{Leff} \PY{o}{=} \PY{n}{L} \PY{o}{\PYZhy{}} \PY{l+m+mi}{2}\PY{o}{*}\PY{n}{Ldn}
    
    \PY{n}{I\PYZus{}dp} \PY{o}{=} \PY{n}{I\PYZus{}ss}\PY{o}{/}\PY{l+m+mi}{2} \PY{o}{+} \PY{l+m+mf}{0.25}\PY{o}{*}\PY{n}{u\PYZus{}n}\PY{o}{*}\PY{n}{C\PYZus{}ox}\PY{o}{*}\PY{p}{(}\PY{n}{W}\PY{o}{/}\PY{n}{L}\PY{p}{)}\PY{o}{*}\PY{n}{V\PYZus{}id}\PY{o}{*}\PY{n}{np}\PY{o}{.}\PY{n}{sqrt}\PY{p}{(}\PY{l+m+mi}{4}\PY{o}{*}\PY{n}{I\PYZus{}ss}\PY{o}{/}\PY{p}{(}\PY{n}{u\PYZus{}n}\PY{o}{*}\PY{n}{C\PYZus{}ox}\PY{o}{*}\PY{p}{(}\PY{n}{W}\PY{o}{/}\PY{n}{L}\PY{p}{)}\PY{p}{)} \PY{o}{\PYZhy{}} \PY{n}{V\PYZus{}id}\PY{o}{*}\PY{o}{*}\PY{l+m+mi}{2}\PY{p}{)}
    \PY{n}{I\PYZus{}dm} \PY{o}{=} \PY{n}{I\PYZus{}ss}\PY{o}{/}\PY{l+m+mi}{2} \PY{o}{\PYZhy{}} \PY{l+m+mf}{0.25}\PY{o}{*}\PY{n}{u\PYZus{}n}\PY{o}{*}\PY{n}{C\PYZus{}ox}\PY{o}{*}\PY{p}{(}\PY{n}{W}\PY{o}{/}\PY{n}{L}\PY{p}{)}\PY{o}{*}\PY{n}{V\PYZus{}id}\PY{o}{*}\PY{n}{np}\PY{o}{.}\PY{n}{sqrt}\PY{p}{(}\PY{l+m+mi}{4}\PY{o}{*}\PY{n}{I\PYZus{}ss}\PY{o}{/}\PY{p}{(}\PY{n}{u\PYZus{}n}\PY{o}{*}\PY{n}{C\PYZus{}ox}\PY{o}{*}\PY{p}{(}\PY{n}{W}\PY{o}{/}\PY{n}{L}\PY{p}{)}\PY{p}{)} \PY{o}{\PYZhy{}} \PY{n}{V\PYZus{}id}\PY{o}{*}\PY{o}{*}\PY{l+m+mi}{2}\PY{p}{)}

    \PY{k}{return} \PY{n}{I\PYZus{}dp}\PY{p}{,} \PY{n}{I\PYZus{}dm}
\end{Verbatim}
\end{tcolorbox}

    \hypertarget{lecture-10---cmos-technology}{%
\section{Lecture 10 - CMOS
Technology}\label{lecture-10---cmos-technology}}

    \hypertarget{announcements}{%
\subsection{Announcements}\label{announcements}}

    \begin{itemize}
\tightlist
\item
  Design Project Phase 2 posted, due Friday March 19

  \begin{itemize}
  \tightlist
  \item
    PDF submission on Canvas
  \end{itemize}
\end{itemize}

    \hypertarget{week-10}{%
\subsection{Week 10}\label{week-10}}

    \begin{itemize}
\tightlist
\item
  Chapter 18 of Razavi (CMOS Procesing Technology)
\item
  Chapter 19 of Razavi (Layout and Packaging)
\end{itemize}

    \hypertarget{overview}{%
\subsection{Overview}\label{overview}}

    \begin{itemize}
\tightlist
\item
  Last time\ldots{}

  \begin{itemize}
  \tightlist
  \item
    Miller compensation of 2-stage OTAs
  \item
    Ahuja compensation
  \item
    Gain-boosted structures
  \end{itemize}
\item
  Today\ldots{}

  \begin{itemize}
  \tightlist
  \item
    Process design rules
  \item
    MOS transistor layout
  \item
    Matching techniques
  \item
    Digital/analog isolation
  \end{itemize}
\end{itemize}

    \hypertarget{mos-transistor-fabrication}{%
\subsection{MOS transistor
fabrication}\label{mos-transistor-fabrication}}

    \begin{itemize}
\item
  Masks are created using photolithography for each layer and the chip
  is constructed layer by layer
\item
  Source, drain and substrate/well ``ties'' are created via ion
  implantation
\item
  \(SiO_2\) is ``grown'' by placing exposed silicon in an oxidizing
  environment at a high temperature (1000 C)
\item
  Circuit connectivity is realized aluminum ``wires'' and contacts
  between vertically-stacked layers of the chip
\item
  Metal and poly layers are deposited via deposition (e.g.~CVD), where
  wafers are put in a furnace filled with a gas containing the desired
  material
\end{itemize}

    \begin{figure}
\centering
\includegraphics{MOS_layout.png}
\caption{MOS\_layout.png}
\end{figure}

    \hypertarget{mos-design-rules}{%
\subsection{MOS design rules}\label{mos-design-rules}}

    \begin{itemize}
\item
  Minimum device \(L\) is a limitation of a given process technology,
  typically indicated by the process node name (e.g.~40nm)
\item
  The minimum value of \(L\) is a photolithography limitation, and has
  historically been halved every 18 months or so (Moore's Law)
\item
  Recently, however, Moore's Law has been slowing down, forcing
  foundries and designers to invent other ways of improving performance
  (e.g.~parallelization)
\end{itemize}

    \begin{figure}
\centering
\includegraphics{MOS_design_rules.png}
\caption{MOS\_design\_rules.png}
\end{figure}

    \hypertarget{mos-transit-frequency}{%
\subsection{MOS transit frequency}\label{mos-transit-frequency}}

    \begin{itemize}
\item
  The utility of the MOS transistor as a gain device depends on its
  capacity to delivery current to a load (with gain)
\item
  That is, to achieve gain (in terms of power, rather than voltage) the
  output current should be higher than the input current
\item
  The frequency at which the input current magnitude becomes equal to
  that of output current is called the transit frequency (\(f_T\))
\end{itemize}

    \begin{figure}
\centering
\includegraphics{transit_frequency.png}
\caption{transit\_frequency.png}
\end{figure}

    \begin{figure}
\centering
\includegraphics{transit_frequency_small_signal.png}
\caption{transit\_frequency\_small\_signal.png}
\end{figure}

\begin{equation}
i_{in} = v_{gs}(sC_{GS}+sC_{GD}) 
\end{equation}

\begin{equation}
i_{out} = g_m v_{gs}
\end{equation}

    \begin{itemize}
\tightlist
\item
  At the transit frequency, the input and output current magnitudes are
  equal
\end{itemize}

\begin{align}
i_{out} &= |i_{in}(f_T)|\\
\\
g_m v_{gs} &= v_{gs}\cdot 2\pi f_T \cdot (C_{GS} + C_{GD}) 
\end{align}

\begin{itemize}
\tightlist
\item
  This results in \(f_T\) given by
\end{itemize}

\begin{equation}
\boxed{ f_T = \dfrac{g_m}{2\pi \cdot (C_{GS} + C_{GD})} }
\end{equation}

    \begin{itemize}
\item
  \(f_T\) is limited by mobility, threshold voltage, \(V_{GS}\), and
  \(L\)
\item
  \(\mu\) is carrier-dependent (electrons, holes), while \(V_{th}\) and
  \(L\) are process-dependent
\item
  The limit of \(V_{GS}\) is (typically) \(V_{DD}\), setting an upper
  bound on device speed for a given process technology
\item
  The scaling of \(L\) via Moore's Law has allowed a consistent
  improvement in performance for decades, but this is slowing due to
  manufacturing technology limitations
\end{itemize}

    \begin{align}
f_T &= \dfrac{g_m}{2\pi \cdot (C_{GS} + C_{GD})} \\
\\
&= \dfrac{\mu C_{ox} \left(\dfrac{W}{L}\right)(V_{GS} - V_{th})}{2\pi (C_{GS} + C_{GD}) } \\
\\
&\approx \dfrac{\mu C_{ox} \left(\dfrac{W}{L}\right)(V_{GS} - V_{th})}{2\pi \cdot 2/3 \cdot WL C_{ox} } \\
\\
&= \boxed{ \dfrac{3}{2}\dfrac{\mu\cdot (V_{GS}-V_{th})}{2\pi L^2} }
\end{align}

    \hypertarget{metal-routing-design-rules}{%
\subsection{Metal routing design
rules}\label{metal-routing-design-rules}}

    \begin{itemize}
\item
  Each design layer has associated minimum values for width
  (\(W_{min1,2}\)) and spacing (\(S_{1,2}\)) between different objects
  on the same layer
\item
  When connecting between layers using a contact, a minimum value of
  overlap (\(OV_{1,2}\)) is required to ensure a reliable connection is
  made
\item
  Minimum values depend on fabrication tolerances, and rule violation
  can result in shorts between wires or breaks in a conductive path (or
  very high local resistance)
\end{itemize}

    \begin{figure}
\centering
\includegraphics{wiring.png}
\caption{wiring.png}
\end{figure}

    \hypertarget{polysilicon-resistors}{%
\subsection{Polysilicon resistors}\label{polysilicon-resistors}}

    \begin{figure}
\centering
\includegraphics{poly_resistor.png}
\caption{poly\_resistor.png}
\end{figure}

    \begin{figure}
\centering
\includegraphics{silicide_blocking.png}
\caption{silicide\_blocking.png}
\end{figure}

    \begin{itemize}
\item
  High resistance values can be achieved by selectively blocking the
  silicide layer that is deposited on top of polysilicon for MOS gates
\item
  Resistance values typically fall in the range of fifty to a few
  hundred \(\Omega / \square\) (pronunced ``ohms per square''), and are
  somewhat variable (\(\pm 20\%\))
\item
  Due to this variability, precision designs must rely on trimming
  (e.g.~through post-processing) or the use of resistance ratios (as in
  opamp feedback networks)
\end{itemize}

    \hypertarget{metal-insulator-metal-mim-capacitors}{%
\subsection{Metal-insulator-metal (MIM)
capacitors}\label{metal-insulator-metal-mim-capacitors}}

    \begin{figure}
\centering
\includegraphics{MIM_capacitor.png}
\caption{MIM\_capacitor.png}
\end{figure}

    \begin{itemize}
\item
  Linear capacitors can be formed using the insulating material between
  metal layers
\item
  Such capacitors are effectively ``parallel-plate'' capacitors, with
  capacitance determined by the overlapping area of the metal layers
  (\(M_7\) and \(M_8\), in the figure) and the distance bewteen them
  (\(C = \epsilon A/d\))
\item
  MIM capacitors are often constructed using metal layers that are
  distant from the substrate to avoid significant ``parasitic''
  capacitance between the bottom plate and ground
\end{itemize}

    \hypertarget{gate-resistance}{%
\subsection{Gate resistance}\label{gate-resistance}}

    \begin{figure}
\centering
\includegraphics{MOS_gate_resistance_noise.png}
\caption{MOS\_gate\_resistance\_noise.png}
\end{figure}

    \begin{figure}
\centering
\includegraphics{MOS_gate_resistance_2D.png}
\caption{MOS\_gate\_resistance\_2D.png}
\end{figure}

    \begin{itemize}
\item
  The resistance of the polysilicon gate, expressed in
  \(\Omega / \square\), contributes to drain current noise through the
  device \(g_m\)
\item
  For devices with large \(W\) (required for high values of \(g_m\)),
  the gate noise can contribute significantly to the drain current noise
\item
  Transistor layout should be performed as to minimize gate resistance
\end{itemize}

    \hypertarget{multi-finger-transistors}{%
\subsection{Multi-finger transistors}\label{multi-finger-transistors}}

    \begin{figure}
\centering
\includegraphics{MOS_fingers.png}
\caption{MOS\_fingers.png}
\end{figure}

    \begin{itemize}
\item
  Connecting multiple transistor ``fingers'' in parallel reduces the
  total gate resistance and noise, in addition to creating a more
  manageable aspect ratio for different geometry constraints
\item
  The use of multi-fingered devices also makes \(W/L\) ratios between
  related devices (e.g.~in a current mirror) easy to realize
\item
  However, this results in additional parasitic capacitance due to the
  larger perimeter of the source and drain regions
\end{itemize}

    \hypertarget{parallel-mos-devices}{%
\subsection{Parallel MOS devices}\label{parallel-mos-devices}}

    \begin{figure}
\centering
\includegraphics{parallel_MOS_devices.png}
\caption{parallel\_MOS\_devices.png}
\end{figure}

    \begin{itemize}
\item
  Transistors may also be formed as multiple parallel devices,
  connecting the source and drain regions on M1 and the poly gates
  directly with additional polysilicon
\item
  Less area-efficient and contributes more parasitic capacitance (due to
  the increased well perimeter), but useful for certain layout schemes
  and matching in the face of process gradients
\end{itemize}

    \hypertarget{process-gradients}{%
\subsection{Process gradients}\label{process-gradients}}

    \begin{figure}
\centering
\includegraphics{process_gradients.png}
\caption{process\_gradients.png}
\end{figure}

    \begin{align}
C_{ox1} &= C_{ox} + C_{ox} + \Delta C_{ox} \\
&= \boxed{ 2C_{ox} + \Delta C_{ox} }\\
\end{align}

    \begin{align}
C_{ox2} &= C_{ox} + 2 \Delta C_{ox} +  C_{ox} + 3 \Delta C_{ox} \\
&= \boxed{ 2C_{ox} + 5 \Delta C_{ox} }
\end{align}

    \begin{itemize}
\item
  In addition to random mismatch (due to local variations in material
  parameters), systematic variations in process parameters (e.g.~doping
  concentration, oxide thickness, etc.) also arise
\item
  Such variations give rise to mismatch between nominally identical
  devices, particularly those located far from each other on chip
\end{itemize}

    \hypertarget{common-centroid-layout}{%
\subsection{Common-centroid layout}\label{common-centroid-layout}}

    \begin{figure}
\centering
\includegraphics{common_centroid.png}
\caption{common\_centroid.png}
\end{figure}

    \begin{align}
C_{ox1} &= C_{ox} + C_{ox} + 3\Delta C_{ox} \\
&= \boxed{ 2C_{ox} + 3\Delta C_{ox} }\\
\end{align}

    \begin{align}
C_{ox2} &= C_{ox} + \Delta C_{ox} +  C_{ox} + 2 \Delta C_{ox} \\
&= \boxed{ 2C_{ox} + 3 \Delta C_{ox} }
\end{align}

    \begin{itemize}
\item
  Common-centroid layout results in each transistor ``seeing'' the same
  net effect of linear process gradients
\item
  By ensuring matched devices are placed about an axis of symmetry, both
  devices experience the same average value of critical parameters
\end{itemize}

    \hypertarget{dummy-transistors}{%
\subsection{Dummy transistors}\label{dummy-transistors}}

    \begin{figure}
\centering
\includegraphics{dummy_transistors.png}
\caption{dummy\_transistors.png}
\end{figure}

    \begin{itemize}
\item
  The addition of ``dummy'' transistors on either side of the devices to
  be matched further increases the symmetry of the environment seen by
  \(M_{1,2}\)
\item
  In this configuration implant shadowing results in the same
  drain/source asymmetry for both transistors, improving their matching
\end{itemize}

    \hypertarget{common-centroid-2d}{%
\subsection{Common-centroid (2D)}\label{common-centroid-2d}}

    \begin{itemize}
\item
  The common-centroid technique can also be applied in 2 dimensions
\item
  2D common centroid layout provides robustness against linear process
  gradients in both \(x\) and \(y\)
\item
  As an example, the matching of \(pnp\) transistors is critical to the
  performance of a bandgap reference
\item
  Using an emitter area ratio of \(N=8\) results in reasonable bias
  current and resistor values (\(V_T \cdot \ln 8 \approx 52 mV\)), and
  lends itself well to a 2D common-centroid layout
\end{itemize}

    \begin{figure}
\centering
\includegraphics{bandgap_layout.png}
\caption{bandgap\_layout.png}
\end{figure}

    \hypertarget{substrate-noise}{%
\subsection{Substrate noise}\label{substrate-noise}}

    \begin{figure}
\centering
\includegraphics{substrate_noise.png}
\caption{substrate\_noise.png}
\end{figure}

    \begin{itemize}
\item
  Transient currents in digital CMOS circuits can couple to the
  substrate through S/D junction capacitance to bulk
\item
  This current results in a voltage drop due to finite substrate
  resistance \(r_{sub}\)
\item
  Even if \(r_{sub}\) is very small, the finite impedance of bond-wire
  inductance (\(L_B\)) will cause a change in the substrate potential,
  resulting in a noise current in \(M_1\) due to the body effect
\end{itemize}

    \hypertarget{guard-rings}{%
\subsection{Guard rings}\label{guard-rings}}

    \begin{figure}
\centering
\includegraphics{guard_ring.png}
\caption{guard\_ring.png}
\end{figure}

    \begin{itemize}
\item
  Guard rings provide a low-impedance path to ground for current flowing
  through the substrate, increasing isolation for sensitive analog
  circuits
\item
  Deep \(n\)-wells provide an added degree of isolation, at the expense
  of increased circuit area
\end{itemize}

    \hypertarget{summary}{%
\subsection{Summary}\label{summary}}

    \begin{itemize}
\item
  To ensure reliable design performance, layout of transistors and
  circuit blocks should adhere to manufacturing design rules
\item
  Multi-finger transistor layout is used to realize reasonable form
  factors for circuit blocks and to reduce gate resistance (at the
  expense of increases parasitic capacitance)
\item
  Common-centroid layout and the use of dummy transistors ensures a
  uniform environment for matched devices and improves matching of
  device characteristics
\item
  Guard rings and deep \(n\)-wells should be used to isolate sensitive
  analog circuits from substrate noise
\end{itemize}


    % Add a bibliography block to the postdoc
    
    
    
\end{document}
